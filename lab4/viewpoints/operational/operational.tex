\section{Operational viewpoint}

The operational viewpoint describes how the EduPal system, particularly the new real-time quizzing feature, will be operated, administered, and supported when it is running in its production environment.


\subsection{Installation and Upgrade}

\begin{itemize}
    \item \textbf{Installation Process:}
          \begin{itemize}
              \item Develop an installation package for the quizzing feature that includes all necessary binaries, configuration files, and dependencies.
              \item Provide a detailed installation guide that includes pre-requisites, step-by-step instructions, and troubleshooting tips.
          \end{itemize}
    \item \textbf{Upgrade Approach:}
          \begin{itemize}
              \item \textbf{Big Bang Upgrade:} Plan for a scheduled downtime where the entire system is upgraded at once. Ensure comprehensive testing in a staging environment before the upgrade.
              \item Include rollback procedures in case the upgrade needs to be reverted due to unforeseen issues.
          \end{itemize}
\end{itemize}


\subsection{Functional Migration}

\begin{itemize}
    \item \textbf{Big Bang Migration:} A single-step migration will be performed during a maintenance window to introduce the new multiple participant quizzing feature.
\end{itemize}


\subsection{Data Migration}

\begin{itemize}
    \item \textbf{Existing Data:}
          As this is a new feature, there is no data related to quizzes that needs to be migrated. However, the system will be designed to handle future data migration requirements.
    \item \textbf{ETL Tools:}
          No ETL tools are necessary, as there is no existing data to extract, transform, or load.
\end{itemize}


\subsection{Operational Monitoring and Control}

\begin{itemize}
    \item \textbf{Monitoring Tools:}
          \begin{itemize}
              \item Implement Prometheus to monitor system performance, resource usage, and user activity in real-time.
          \end{itemize}
    \item \textbf{Control Operations:}
          \begin{itemize}
              \item Provide an admin interface for managing quiz sessions, including starting, pausing, and stopping quizzes.
              \item Ensure logs are generated for key operations and stored centrally for audit and troubleshooting purposes.
          \end{itemize}
\end{itemize}


\subsection{Alerting}

\begin{itemize}
    \item \textbf{Alert Rules:}
          \begin{itemize}
              \item \textbf{System Outages:}
                    Alert when any major components of the system (database, object storage system) go offline. Trigger alerts immediately upon detecting downtime.
              \item \textbf{Performance:}
                    Set up alerts to be fired for high CPU or memory usage exceeding 85\% for more than 3 minutes. Alert on prolonged response times. Monitor and alert on unusual spikes in network latency.
              \item \textbf{Failed Quizzes:}
                    Alert when a there are more than 5 crashed or otherwise failed quiz sessions in 10 minutes.
              \item \textbf{Security Events:}
                    Alert on multiple failed login attempts from the same IP address, indicating potential brute-force attacks.
                    Trigger alerts for unauthorized access attempts or changes to critical system configurations.
              \item \textbf{Other Critical Events:}
                    Alert on application errors and exceptions that are not handled gracefully by the system.
          \end{itemize}
    \item \textbf{Alert Management:}
          \begin{itemize}
              \item Integrate with PagerDuty to ensure alerts are routed to the relevant personnel based on severity and nature of the incident.
              \item Configure PagerDuty to escalate alerts if not acknowledged within a predefined timeframe.
              \item Maintain a log of all incidents, their resolution status, and any follow-up actions required.
              \item Periodically review alert thresholds and rules to ensure they remain effective and relevant.
          \end{itemize}
\end{itemize}


\subsection{Configuration Management}

\begin{itemize}
    \item \textbf{Centralized Management:}
          \begin{itemize}
              \item Store configuration files in a version-controlled repository (GitLab) and use Ansible - a configuration management tool, to apply configurations consistently across environments.
          \end{itemize}
    \item \textbf{Parameter Management:}
          \begin{itemize}
              \item Maintain environment-specific configuration files for different stages (development, staging, production).
              \item Ensure sensitive configuration parameters (e.g., API keys, database credentials) are stored securely, using HashiCorp Vault.
          \end{itemize}
\end{itemize}


\subsection{Performance Monitoring}

\begin{itemize}
    \item \textbf{Performance Metrics:}
          \begin{itemize}
              \item Track key metrics: response times, system load, CPU/memory usage, network latency, and error rates.
              \item Track the number of active users and concurrent quiz sessions to identify peak usage times.
              \item Use APM tools for detailed performance insights.
          \end{itemize}
    \item \textbf{Scalability:}
          \begin{itemize}
              \item Implement horizontal scaling and use load balancers to handle increased loads.
              \item Use auto-scaling to adjust resources based on demand.
          \end{itemize}
\end{itemize}


\subsection{Support}

\begin{itemize}
    \item \textbf{Support Channels:}
          \begin{itemize}
              \item Provide email support, ticketing system, for users to report issues and request assistance.
              \item Ensure support staff are trained on the new quizzing feature and have access to internal documentation.
          \end{itemize}
    \item \textbf{Documentation:}
          \begin{itemize}
              \item Create and maintain comprehensive user guides, FAQs, and troubleshooting documents.
              \item Update internal knowledge bases to include information on the quizzing feature.
          \end{itemize}
\end{itemize}


\subsection{Backup and Restore}

\begin{itemize}
    \item \textbf{Data Backup:}
          \begin{itemize}
              \item Use cloud configurations to schedule regular backups of quiz-related data, including questions, answers, and user results.
              \item Use a combination of full and incremental backups to balance between redundancy and resource usage.
          \end{itemize}
    \item \textbf{Restore Procedures:}
          \begin{itemize}
              \item Develop and document cloud-based procedures for restoring quiz data from backups.
              \item Regularly test restore procedures to ensure they work as expected in the event of data loss.
          \end{itemize}
\end{itemize}


\subsection{Operation in Third-Party Environments}

\begin{itemize}
    \item \textbf{Cloud Compatibility:}
          \begin{itemize}
              \item Ensure the quizzing feature is compatible with cloud environments like AWS, Azure, or Google Cloud.
              \item Use cloud-native services for scalability, such as managed databases and load balancers.
          \end{itemize}
    \item \textbf{Integration:}
          \begin{itemize}
              \item Integrate with third-party monitoring and alerting tools.
              \item Ensure compliance with third-party SLA (Service Level Agreement) requirements.
          \end{itemize}
\end{itemize}


\subsection{Stakeholders}

\begin{itemize}
    \item \textbf{System Administrators:}
          \begin{itemize}
              \item Ensure the new feature is installed, configured, and maintained properly.
              \item Monitor the system's performance and handle alerts promptly.
          \end{itemize}
    \item \textbf{Production Engineers:}
          \begin{itemize}
              \item Oversee the deployment and integration of the quizzing feature.
              \item Ensure that the system scales according to demand and performance requirements.
          \end{itemize}
    \item \textbf{Developers:}
          \begin{itemize}
              \item Develop the quizzing feature to be easily installable and upgradeable.
              \item Provide automation scripts for data migration and configuration management.
          \end{itemize}
    \item \textbf{Testers:}
          \begin{itemize}
              \item Conduct extensive testing to ensure the feature works as intended under various conditions.
              \item Validate backup and restore procedures to ensure data integrity.
          \end{itemize}
    \item \textbf{Communicators:}
          \begin{itemize}
              \item Clearly communicate about changes and new features to users.
              \item Provide training materials, release notes, and support documentation.
          \end{itemize}
    \item \textbf{Assessors:}
          \begin{itemize}
              \item Evaluate the operational readiness of the quizzing feature.
              \item Ensure compliance with internal standards and external regulations, such as GDPR for data protection.
          \end{itemize}
\end{itemize}


\subsection{Security Perspective}

\begin{itemize}
    \item \textbf{System Security:} Implement regular security assessments and audits to ensure the operational integrity of the quizzing system.
    \item \textbf{Access Control:} Ensure proper access controls are in place to manage user roles and permissions within the system.
\end{itemize}


\subsection{Performance and Scalability}

The operational readiness of the real-time quizzing feature requires performance monitoring and scalability measures. Performance metrics such as response times, system load, and resource usage will be continually tracked. Scalability will be addressed by implementing horizontal or vertical scaling for backend services and using load balancers to evenly distribute user traffic, ensuring the system can handle increased loads during peak quiz times without degradation in user experience.


\subsection{Availability and Recovery}

High availability and quick recovery are critical for the real-time quizzing feature. Redundant systems and failover mechanisms will be implemented to minimize downtime. Regular backups will be performed, and restore procedures will be documented and tested to ensure data can be recovered swiftly in case of a failure. Additionally, using cloud services with built-in high availability features will further enhance system resilience.


\subsection{Evolution Perspective}

To accommodate future enhancements and changes, the operational processes will include provisions for continuous integration and continuous deployment. This will allow for seamless updates and improvements to the quizzing feature with minimal disruption. Regular feedback from users and stakeholders will make iterative development and operational adjustments.


\subsection{Location Perspective}

Unrelated.


\subsection{Regulation Perspective}

Operational compliance with relevant regulations, such as data protection, will be maintained. Regular audits and assessments will ensure that the quizzing feature adheres to these regulations, with any necessary operational adjustments made promptly to remain compliant.


\subsection{Usability Perspective}

From an operational standpoint, ensuring usability involves regular monitoring and analysis of user interactions to identify and resolve any operational bottlenecks or issues. Providing comprehensive documentation, intuitive admin interfaces, and responsive support channels will ensure that both users and administrators can effectively utilize and manage the quizzing feature, maintaining a high level of user satisfaction.
