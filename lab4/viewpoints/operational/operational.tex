\section{Operational viewpoint}

The operational viewpoint describes how the EduPal system, particularly the new real-time quizzing feature, will be operated, administered, and supported when it is running in its production environment.


\subsection{Installation and Upgrade}

\begin{itemize}
    \item \textbf{Installation Process:}
          \begin{itemize}
              \item Develop an installation package for the quizzing feature that includes all necessary binaries, configuration files, and dependencies.
              \item Provide a detailed installation guide that includes pre-requisites, step-by-step instructions, and troubleshooting tips.
          \end{itemize}
    \item \textbf{Upgrade Approach:}
          \begin{itemize}
              \item \textbf{Big Bang Upgrade:} Plan for a scheduled downtime where the entire system is upgraded at once. Ensure comprehensive testing in a staging environment before the upgrade.
              \item \textbf{Incremental Upgrade:} Deploy the quizzing feature to a subset of users first (canary deployment) to identify and resolve issues before a full rollout.
              \item Include rollback procedures in case the upgrade needs to be reverted due to unforeseen issues.
          \end{itemize}
\end{itemize}


\subsection{Functional Migration}

\begin{itemize}
    \item \textbf{Big Bang Migration:} If applicable, perform a single-step migration during a maintenance window to switch from any existing quizzing mechanisms to the new feature.
    \item \textbf{Parallel Run:} Optionally, run both the old and new quizzing systems concurrently. This allows users to transition gradually and provides a fallback if issues arise with the new feature.
\end{itemize}


\subsection{Data Migration}

\begin{itemize}
    \item \textbf{Existing Data:}
          \begin{itemize}
              \item Identify any existing quiz data that needs to be migrated.
              \item Develop scripts to extract, transform, and load (ETL) this data into the new quizzing system’s format.
          \end{itemize}
    \item \textbf{ETL Tools:}
          \begin{itemize}
              \item Use ETL tools such as Talend or Apache NiFi for automating data migration processes.
              \item Test data migration in a staging environment to ensure accuracy and completeness.
          \end{itemize}
\end{itemize}


\subsection{Operational Monitoring and Control}

\begin{itemize}
    \item \textbf{Monitoring Tools:}
          \begin{itemize}
              \item Implement tools such as Prometheus, Grafana, or Datadog to monitor system performance, resource usage, and user activity in real-time.
          \end{itemize}
    \item \textbf{Control Operations:}
          \begin{itemize}
              \item Provide an admin interface for managing quiz sessions, including starting, pausing, and stopping quizzes.
              \item Ensure logs are generated for key operations and stored centrally for audit and troubleshooting purposes.
          \end{itemize}
\end{itemize}


\subsection{Alerting}

\begin{itemize}
    \item \textbf{Alerts Configuration:}
          \begin{itemize}
              \item Set up alerts for critical events such as system outages, performance degradation, and failed quiz submissions.
              \item Use tools like PagerDuty or Opsgenie to ensure alerts reach the relevant personnel promptly.
          \end{itemize}
    \item \textbf{Alert Management:}
          \begin{itemize}
              \item Integrate alerts with a central console (e.g., ServiceNow) to manage and track incidents.
              \item Define escalation policies to ensure timely response to unresolved alerts.
          \end{itemize}
\end{itemize}


\subsection{Configuration Management}

\begin{itemize}
    \item \textbf{Centralized Management:}
          \begin{itemize}
              \item Store configuration files in a version-controlled repository (GitLab) and use a configuration management tool like Ansible, Puppet, or Chef to apply configurations consistently across environments.
          \end{itemize}
    \item \textbf{Parameter Management:}
          \begin{itemize}
              \item Maintain environment-specific configuration files for different stages (development, staging, production).
              \item Ensure sensitive configuration parameters (e.g., API keys, database credentials) are stored securely, using tools like HashiCorp Vault.
          \end{itemize}
\end{itemize}


\subsection{Performance Monitoring}

\begin{itemize}
    \item \textbf{Performance Metrics:}
          \begin{itemize}
              \item Track metrics such as response times, system load, CPU/memory usage, and network latency.
              \item Use APM (Application Performance Management) tools like New Relic or AppDynamics to get detailed insights into application performance.
          \end{itemize}
    \item \textbf{Scalability:}
          \begin{itemize}
              \item Implement horizontal scaling for the backend to handle increased loads during peak quiz times.
              \item Use load balancers to distribute traffic evenly across instances.
          \end{itemize}
\end{itemize}


\subsection{Support}

\begin{itemize}
    \item \textbf{Support Channels:}
          \begin{itemize}
              \item Provide some sort of support channel, such as email, ticketing system, for users to report issues and request assistance.
              \item Ensure support staff are trained on the new quizzing feature and have access to internal documentation.
          \end{itemize}
    \item \textbf{Documentation:}
          \begin{itemize}
              \item Create and maintain comprehensive user guides, FAQs, and troubleshooting documents.
              \item Update internal knowledge bases to include information on the quizzing feature.
          \end{itemize}
\end{itemize}


\subsection{Backup and Restore}

\begin{itemize}
    \item \textbf{Data Backup:}
          \begin{itemize}
              \item Schedule regular backups of quiz-related data, including questions, answers, and user results.
              \item Use a combination of full and incremental backups to balance between redundancy and resource usage.
          \end{itemize}
    \item \textbf{Restore Procedures:}
          \begin{itemize}
              \item Develop and document procedures for restoring quiz data from backups.
              \item Regularly test restore procedures to ensure they work as expected in the event of data loss.
          \end{itemize}
\end{itemize}


\subsection{Operation in Third-Party Environments}

\begin{itemize}
    \item \textbf{Cloud Compatibility:}
          \begin{itemize}
              \item Ensure the quizzing feature is compatible with cloud environments like AWS, Azure, or Google Cloud.
              \item Use cloud-native services for scalability, such as managed databases and load balancers.
          \end{itemize}
    \item \textbf{Integration:}
          \begin{itemize}
              \item Integrate with third-party monitoring and alerting tools if hosted externally.
              \item Ensure compliance with third-party SLA (Service Level Agreement) requirements.
          \end{itemize}
\end{itemize}


\subsection{Stakeholders}

\begin{itemize}
    \item \textbf{System Administrators:}
          \begin{itemize}
              \item Ensure the new feature is installed, configured, and maintained properly.
              \item Monitor the system’s performance and handle alerts promptly.
          \end{itemize}
    \item \textbf{Production Engineers:}
          \begin{itemize}
              \item Oversee the deployment and integration of the quizzing feature.
              \item Ensure that the system scales according to demand and performance requirements.
          \end{itemize}
    \item \textbf{Developers:}
          \begin{itemize}
              \item Develop the quizzing feature to be easily installable and upgradeable.
              \item Provide automation scripts for data migration and configuration management.
          \end{itemize}
    \item \textbf{Testers:}
          \begin{itemize}
              \item Conduct extensive testing to ensure the feature works as intended under various conditions.
              \item Validate backup and restore procedures to ensure data integrity.
          \end{itemize}
    \item \textbf{Communicators:}
          \begin{itemize}
              \item Facilitate clear communication about changes and new features to users.
              \item Provide training materials, release notes, and support documentation.
          \end{itemize}
    \item \textbf{Assessors:}
          \begin{itemize}
              \item Evaluate the operational readiness of the quizzing feature.
              \item Ensure compliance with internal standards and external regulations, such as GDPR for data protection.
          \end{itemize}
\end{itemize}


\subsection{Security Perspective}

\begin{itemize}
    \item \textbf{System Security:} Implement regular security assessments and audits to ensure the operational integrity of the quizzing system.
    \item \textbf{Access Control:} Ensure proper access controls are in place to manage user roles and permissions within the system.
\end{itemize}


\subsection{Performance and Scalability}

The operational readiness of the real-time quizzing feature requires robust performance monitoring and scalability measures. Performance metrics such as response times, system load, and resource usage will be continually tracked using APM tools like New Relic or AppDynamics. Scalability will be addressed by implementing horizontal scaling for backend services and using load balancers to evenly distribute user traffic, ensuring the system can handle increased loads during peak quiz times without degradation in user experience.


\subsection{Availability and Recovery}

High availability and quick recovery are critical for the real-time quizzing feature. Redundant systems and failover mechanisms will be implemented to minimize downtime. Regular backups will be performed, and restore procedures will be documented and tested to ensure data can be recovered swiftly in case of a failure. Additionally, using cloud services with built-in high availability features will further enhance system resilience.


\subsection{Evolution Perspective}

To accommodate future enhancements and changes, the operational processes will include provisions for continuous integration and continuous deployment (CI/CD). This will allow for seamless updates and improvements to the quizzing feature with minimal disruption. Regular feedback from users and stakeholders will inform iterative development and operational adjustments.


\subsection{Location Perspective}

Unrelated.


\subsection{Regulation Perspective}

Operational compliance with relevant regulations, such as data protection, will be maintained. Regular audits and assessments will ensure that the quizzing feature adheres to these regulations, with any necessary operational adjustments made promptly to remain compliant.


\subsection{Usability Perspective}

From an operational standpoint, ensuring usability involves regular monitoring and analysis of user interactions to identify and resolve any operational bottlenecks or issues. Providing comprehensive documentation, intuitive admin interfaces, and responsive support channels will ensure that both users and administrators can effectively utilize and manage the quizzing feature, maintaining a high level of user satisfaction.
