\section{Development View}

\subsection{Module Organization}

The EduPal quiz integration feature is organized into several interconnected modules, each responsible for specific aspects of the system:

\subsubsection{User Interface Module}

\begin{itemize}
  \item \textbf{QuizPage}: The main page where users can start the quiz.
  \item \textbf{QuestionPage}: The page that displays quiz questions to the users.
  \item \textbf{ResultPage}: The page that shows the results after the quiz is completed.
\end{itemize}

\subsubsection{Quiz Session Module}

\begin{itemize}
  \item \textbf{QuizSessionController}: Manages user interactions during a quiz session.
  \item \textbf{QuizSessionService}: Handles the business logic associated with a quiz session.
  \item \textbf{QuizSessionRepository}: Manages the data storage and retrieval for quiz sessions.
\end{itemize}

\subsubsection{Quiz Management Module}

\begin{itemize}
  \item \textbf{QuizController}: Manages the interactions related to quiz management.
  \item \textbf{QuizService}: Contains the business logic for quiz management.
  \item \textbf{QuizRepository}: Responsible for storing and retrieving quiz-related data.
\end{itemize}

\subsubsection{Question Bank Module}

\begin{itemize}
  \item \textbf{QuestionController}: Manages interactions related to question management.
  \item \textbf{QuestionService}: Contains the business logic for question management.
  \item \textbf{QuestionRepository}: Responsible for storing and retrieving question data.
\end{itemize}

\subsubsection{Common Processing}

\item \textbf{Data Validation:} Uses React with Formik and Yup for form handling and schema-based validation.

\item \textbf{Error Handling:} Uses ASP.NET Core middleware for global exception handling, logging errors to a centralized logging service (like Serilog) and returning standardized error responses.

\item \textbf{User Authentication:} Manages user authentication and authorization using JWT tokens for securing API endpoints. Middleware checks token validity and user roles before processing quiz-related requests.

\begin{figure}[htbp]
    \centering
    \includedrawio[width=\textwidth]{viewpoints/development/development.drawio}
    \caption{Module Organization Diagram}
    \label{fig:development}
\end{figure}

\subsection{Standardization of Design}

\subsubsection{Design Patterns}

\begin{itemize}
  \item \textbf{Factory Pattern:} Used for creating instances of different quiz types.
  \item \textbf{Repository Pattern:} Abstracts data access, promoting a clean separation between the business logic and data access layers.
  \item \textbf{Singleton Pattern:} Ensures a single instance of services like QuizSessionService to manage state across the application.
\end{itemize}

\subsubsection{Standard UI Components}

  \item \textbf{Component Library:} Uses a custom component library based on Material-UI. Standard components like Button, Card, and Modal are reused across all quiz-related pages to maintain a consistent look and feel.

\subsubsection{Consistent User Flows}

\item \textbf{Quiz Taking Flow:} Users navigate from the quiz list to the quiz page, answer questions sequentially, and then view results. Each step is guided by consistent visual cues and navigation buttons.

\item \textbf{Quiz Management Flow:} Admins follow a flow from creating a new quiz, adding questions, setting correct answers, and publishing the quiz.

\subsection{Standardization of Testing}

\subsubsection{Unit Testing}

\item \textbf{Frontend:} Uses Jest and React Testing Library to test individual React components. Tests cover rendering, user interactions, and state changes.

\item \textbf{Backend:} Utilizes NUnit for testing ASP.NET Core services and repositories. Tests include validation logic, business rules, and data access methods.

\subsubsection{Integration Testing}

\item \textbf{API Endpoints:} Uses Postman to automate testing of API endpoints.

\item \textbf{Frontend-Backend Integration:} End-to-end tests with Cypress to verify that the frontend and backend interact correctly.

\subsection{Instrumentation}

\item \textbf{Monitoring Performance and Usage:}

\item \textbf{Logging:} Incorporates Serilog for structured logging. Logs include details on user interactions with quizzes, errors, and performance metrics.

\item \textbf{Analytics:} Uses Google Analytics to track user behavior, quiz engagement rates, and completion statistics. Custom events track specific actions like quiz start, question answer, and quiz completion.

\item \textbf{Usage Metrics:} Stores data on question difficulty, user scores, and completion times in a PostgreSQL database. Regular reports are generated to identify trends and areas for improvement.

\subsection{Codeline Organization}

\textbf{Modular Structure:}

\item \textbf{Frontend:} Organized into directories for components (/components), pages (/pages), services (/services), and utilities (/utils). Each component and page related to the quiz feature has its own directory.

\item \textbf{Backend:} Follows a similar structure with directories for controllers (/Controllers), services (/Services), repositories (/Repositories), and models (/Models). The quiz feature-specific code is encapsulated within its own subdirectories.

\item \textbf{Branching Strategy:} Uses Git with a feature branch workflow.

\subsection{Stakeholder Concerns}

\subsubsection{Developers}

\begin{itemize}
  \item Need clear module boundaries and documentation to facilitate development and maintenance.
  \item Require standardized testing and instrumentation practices to ensure code quality and performance.
\end{itemize}

\subsubsection{Production Engineers}

\begin{itemize}
  \item Concerned with the reliability and scalability of the quiz feature.
  \item Need comprehensive logging and monitoring to quickly identify and resolve issues.
\end{itemize}

\subsubsection{Testers}

\begin{itemize}
  \item Require detailed test plans and automated testing scripts to efficiently validate the system.
  \item Benefit from standardized testing approaches to ensure consistent test coverage and results.
\end{itemize}

\end{document}
