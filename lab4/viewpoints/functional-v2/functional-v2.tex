\section{Functional viewpoint}

\subsection{Functional capabilities}

Based on the requirements presented in Context section, the EduPal must support the following core capabilities:

\begin{enumerate}
  \item CRUD operations on the quiz
  \item Real time quiz rooms with multiple players, checking answers and determining the winner
\end{enumerate}

\subsection{External interfaces}

\subsubsection{User Web Browser}

External actor responsible for communication with front-facing EduPal interfaces as well as displaying EduPal web pages. Web browser, being external and user controlled entity, is not trusted and its requests are verified by front-facing interfaces.

The communication is done via two channels : HTTP REST API and SignalR sockets. Most of the CRUD operations are done via REST API and the control flow is inbound - EduPal receives a request, checks whether it is valid and performs the required action yielding a response. Real time functionality (real time quiz rooms in our case) or functionality requiring the system to notify the web browser is performed using SignalR sockets. Data can flow in any direction and system and browser may not respond to certain requests. Communication is always initiated by the browser.

\subsubsection{Object Data Storage}

External system that is responsible for storing data objects and is used for storing question images. Data Flow is outbound - EduPal requests the system, to store, update or delete the file or permissions and requests an URI that is either passed to the user web browser, or processed by EduPal. Communication method is dependant on external system - EduPal must have a component responsible for performing the communication and exposing the interface to the rest of internal components.

\subsection{Internal structure}

EduPal is a single monolithic entity consisting of loosely coupled components that interact with each other using defined interfaces.

\subsection{Components}

\begin{figure}[H]
  \includedrawio[width=\textwidth]{viewpoints/functional-v2/components.drawio}
  \label{fig:viewpoint-functional-components-v2}
  \caption{Component diagram}
\end{figure}

\begin{description}
  \item[User Web Browser:] Component responsible for communication with front-facing EduPal interfaces as well as displaying EduPal web pages. Web browser, being external and user controlled entity, is not trusted and its requests are verified by front-facing interfaces.
  \item[React Web Page Server:] Component responsible for serving EduPal React webpages to the web browser. This component does not have access to other parts of the EduPal system and is only responsible for delivery of hypertext, media and code assets that are executed and rendered on the web browser.
  \item[EduPal Web Interface:] Component responsible for handling user requests, validating them and communicating with functionality related components (CRUD Handler for data fetching and mutation and Quiz Game Handler for quiz game) to perform the operation and prepare the response.
  \item[Auhtorization Subsystem:] Component responsible for determining whether certain type of request can be processed (checks whether user has the privilege) and whether user has the right to perform certain operations (such as quiz creation).
  \item[CRUD Handler:] This component offer creation, update, delete and other data management commands and are responsible for checking if operation can be performed based on business rules.
  \item[Object Data Storage:] External component responsible for object storage. 
  \item[Image Storage:] Component responsible for communication with the data object storage system in order to store or retrieve the data.
  \item[Data Storage components:] Components responsible for abstracting data access and update logic. Quiz Game Session Data Storage is extracted as a separate component as a separation of concerns because quiz game sessions are short, require frequent updates and are unrelated to the rest of EduPal system.
  \item[Quiz Game Session Handler:] Component responsible for handling the quiz game business logic: grading answers, generating room code and so on.
\end{description}

\subsection{Traceability}

\begin{figure}[H]
  \label{tab:functional-viewpoint-component-traceability-v2}
  \centering
  \begin{adjustbox}{max width=\textwidth}
      \begin{tabular}{|l|l|l|l|l|l|l|}
      \hline
          \textbf{Component/Requirement} & \textbf{FR1} & \textbf{FR2} & \textbf{FR3} & \textbf{FR4-6,8-10} & \textbf{FR7} & \textbf{FR11-17} \\ \hline
          \textbf{React Web Page Server} & X & ~ & ~ & X & X & X \\ \hline
          \textbf{EduPal Web Interface} & X & ~ & ~ & X & ~ & X \\ \hline
          \textbf{Auth Subsystem} & ~ & X & X & X & ~ & X \\ \hline
          \textbf{CRUD Handler} & X & ~ & ~ & X & ~ & X \\ \hline
          \textbf{Quiz Game Session Handler} & ~ & ~ & ~ & ~ & ~ & X \\ \hline
          \textbf{Data Catalog} & X & ~ & ~ & X & ~ & X \\ \hline
          \textbf{Image Storage} & ~ & ~ & ~ & X & ~ & ~ \\ \hline
      \end{tabular}
\end{adjustbox}
  \caption{Traceability between components and requirements}
\end{figure}

\subsection{Security Perspective}

\subsubsection{Resources}

System resource access is performed by CRUD Handler that communicates with Auth subsystem to determine if operation can be performed by the current user.

\subsubsection{Principle auth}

EduPal Web Interface component is responsible for secure exchange of authentication tokens that Auth subsystem uses to determine, who is the user that performs the action and what privileges they have depending on the operation.

\subsubsection{Policies}

Auth subsystem will deny requests as unauthorized in the following cases:

\begin{enumerate}
  \item Any operation by unauthenticated user, except login and register.
  \item Quiz modification operations performed by the user, who is not owner of the topic.
\end{enumerate}

\subsubsection{Threats}

Since EduPal does not have high security requirements there are less possible threats. Attacker may take advantage of of lack of security processing of certain operations (e.g. developer's mistake) to do unauthorized action.

\subsubsection{Availability}

EduPal Web Interface component is responsible for rate limiting of the connections.

\subsection{Performance and scalability perspective}

Data catalog component may use caching to speed up database lookup. CRUD Handler can process multiple requests simultaneously. However, a monolithic architecture heavily limits the throughput and does not scale well, since it is impossible to scale horizontally (and vertical scaling has an upper bound). Current architecture is acceptable since non-functional requirements are not demanding and other concerns (such as time to implement new features and data consistency) are more important.

\subsection{Availability and resilience perspective}

No functional changes are required since there are no system availability functional requirements (such as offline mode).

\subsection{Evolution perspective}

Layered model ensures that it is easy to add new functionality or change existing one.

\subsection{Location perspective}

Unrelated.

\subsection{Regulation perspective}

Due to GDPR and other law enforcements, EduPal must ensure that business logic services only retrieve and store the data that is absolutely required for operation.

\subsection{Usability perspective}

EduPal Web Interface must expose convenient API that does not require to do multiple requests. For example, it is considered inconvenient if adding 2 questions to a quiz requires 2 API calls, instead system must quiz edit API must be flexible and allow complex state changes within single request.
