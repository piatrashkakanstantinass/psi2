\section{Context viewpoint}


\subsection{System Scope and Responsibilities}

\subsection*{Scope}
EduPal is a learning aid platform for students and teachers. It allows educators to upload course information, and students to access learning material and tools.

\begin{itemize}
    \item Pomodoro Timer: Helps students manage study sessions effectively.
    \item Note Creation and Export: Allows students to create notes and export them in PDF format.
    \item Goal Setting: Enables users to set and track educational goals.
    \item ChatGPT Integration: Provides an AI-powered assistant to help answer questions and provide learning support.
    \item Subject and Topic Management: Users can organize their studies by subjects and topics, and upload conspectuses.
    \item Real-Time Quizzing: This will allow users to create quiz rooms, share access codes, and participate in real-time quizzes, with results displayed immediately.
\end{itemize}


\subsection{Identity of External Entities and Services and Data Used}

\subsection*{External Entities}
\begin{itemize}
    \item \textbf{Users (Students and Teachers):} Interact with EduPal to upload, manage, and access course content or take quizzes, utilise the features such as Pomodoro timer, quizzes, etc.
    \item \textbf{PostgreSQL Database:} The central data storage system for EduPal. Stores all application data, except conspectus files.
    \item \textbf{ChatGPT API:} External service used for providing AI-based assistance and responses within the platform.
\end{itemize}

\subsection*{Data Used}
\begin{itemize}
    \item \textbf{User Data:} Includes personal information, preferences, learning progress, and quiz results.
    \item \textbf{Course Content and upvotes:} Uploaded by users, includes any kind of conspectuses and conspectus ratings.
    \item \textbf{Notes and Goals:} Created by users, includes topic notes, goal-setting data, and progress tracking.
    \item \textbf{Quiz Data:} Includes questions, answers, and results of the real-time quizzes.
    \item \textbf{ChatGPT Interaction Data:} Includes user queries and responses generated by ChatGPT.
\end{itemize}

Two versions for this concern

\subsection{Nature and Characteristics of External Entities}

\subsection*{Users}
\begin{itemize}
    \item Require a reliable and user-friendly interface.
    \item Access the system from various devices and locations.
    \item Depend on system availability and performance for effective learning.
\end{itemize}

\subsection*{PostgreSQL Database}
\begin{itemize}
    \item High availability and reliability.
    \item Capable of handling large amounts of data.
\end{itemize}

\subsection*{ChatGPT API}
\begin{itemize}
    \item Requires Stable and Secure Connectivity.
    \item Efficient Handling of API Requests.
    \item Compliance with GDPR.
\end{itemize}


\subsection{Nature and Characteristics of External Entities}

\subsection*{Users}
\begin{itemize}
    \item \textbf{Require a Reliable and User-Friendly Interface:}
          \begin{itemize}
              \item The interface must be intuitive, easy to navigate, and responsive to ensure a positive user experience. This includes clear instructions, accessible features, and a visually appealing design.
          \end{itemize}
    \item \textbf{Access the System from Various Devices and Locations:}
          \begin{itemize}
              \item Users should be able to access EduPal from desktops, laptops, tablets, and smartphones. The system needs to be compatible with various operating systems and web browsers. It should also handle different internet connection speeds efficiently, ensuring functionality remains consistent.
          \end{itemize}
    \item \textbf{Depend on System Availability and Performance for Effective Learning:}
          \begin{itemize}
              \item The platform must be highly available with minimal downtime to support continuous learning. Performance is somewhat important, as slower load times or unresponsive features can disrupt the learning process and reduce user satisfaction.
          \end{itemize}
\end{itemize}

\subsection*{PostgreSQL Database}
\begin{itemize}
    \item \textbf{High Availability and Reliability:}
          \begin{itemize}
              \item The database should have mechanisms for redundancy, failover, and disaster recovery to ensure data is always accessible and safe from loss or corruption.
          \end{itemize}
    \item \textbf{Capable of Handling Large Amounts of Data:}
          \begin{itemize}
              \item The database should be scalable to accommodate growing data volumes as more users join and contribute content.
          \end{itemize}
\end{itemize}

\subsection*{ChatGPT API}
\begin{itemize}
    \item \textbf{Requires Stable and Secure Connectivity:}
          \begin{itemize}
              \item The integration with ChatGPT needs stable and secure internet connectivity to function correctly, ensuring user queries are processed without interruption.
          \end{itemize}
    \item \textbf{Efficient Handling of API Requests and Responses:}
          \begin{itemize}
              \item The system should efficiently manage API requests to ChatGPT to maintain performance and provide timely responses to users.
          \end{itemize}
    \item \textbf{Compliance with GDPR:}
          \begin{itemize}
              \item Interaction data between users and ChatGPT should be handled according to GDPR.
          \end{itemize}
\end{itemize}


\subsection{Identity and Responsibilities of External Interfaces}

\subsection*{External Interfaces}
\begin{itemize}
    \item \textbf{User Interface:}
          \begin{itemize}
              \item \textbf{Identity:} The web-based interface accessed by users (students and teachers) via browsers on various devices.
              \item \textbf{Responsibilities:}
                    \begin{itemize}
                        \item Provide an intuitive, easy-to-navigate, and responsive platform for interacting with EduPal features.
                        \item Allow users to upload, manage, and access course content, take quizzes, create and export notes, and set goals, track study time with the Pomodoro timer.
                        \item Ensure compatibility with different devices and browsers, maintaining consistent functionality and performance.
                    \end{itemize}
          \end{itemize}
    \item \textbf{PostgreSQL Database Interface:}
          \begin{itemize}
              \item \textbf{Identity:} The interface through which the backend communicates with the PostgreSQL database.
              \item \textbf{Responsibilities:}
                    \begin{itemize}
                        \item Store and retrieve all application data, except conspectus files.
                        \item Ensure high availability, reliability, and data integrity.
                        \item Scale to accommodate growing data volumes.
                    \end{itemize}
          \end{itemize}
    \item \textbf{ChatGPT API:}
          \begin{itemize}
              \item \textbf{Identity:} The external interface that connects EduPal with the ChatGPT service.
              \item \textbf{Responsibilities:}
                    \begin{itemize}
                        \item Provide AI assistance and responses to user queries within the platform.
                        \item Maintain stable and secure connectivity to ensure uninterrupted service.
                        \item Manage API responses efficiently to provide timely responses.
                        \item Comply with GDPR.
                    \end{itemize}
          \end{itemize}
\end{itemize}


\subsection{Nature and Characteristics of External Interfaces}

\subsection*{User Interface}
\begin{itemize}
    \item \textbf{Expected Volumes:}
          \begin{itemize}
              \item High volume of requests during peak hours such as during exam periods or assignment deadlines.
              \item Varying data sizes, from small interactions (e.g., note-taking) to larger uploads (e.g., course material).
              \item Growth expected as user base increases.
          \end{itemize}
    \item \textbf{Interaction Types:}
          \begin{itemize}
              \item Mostly ad hoc interactions initiated by users.
              \item Automatic topic archival is scheduled.
          \end{itemize}
    \item \textbf{Interaction Automation Level:}
          \begin{itemize}
              \item Almost all interactions are fully manual, except some that are in between manual and automated, such as the Pomodoro timer.
          \end{itemize}
    \item \textbf{Transactional Nature:}
          \begin{itemize}
              \item None of the interactions are fully transactional, as all of them can be cancelled.
          \end{itemize}
    \item \textbf{Criticality and Timeliness:}
          \begin{itemize}
              \item Timely responses critical for a smooth user experience.
          \end{itemize}
    \item \textbf{Interaction Mode:}
          \begin{itemize}
              \item Primarily message-based, with occasional batch operations (e.g., bulk uploads).
          \end{itemize}
    \item \textbf{Security Requirements:}
          \begin{itemize}
              \item High level of security required, including user authentication and data confidentiality.
          \end{itemize}
    \item \textbf{Service Levels:}
          \begin{itemize}
              \item High availability and low latency expected to ensure user satisfaction.
          \end{itemize}
    \item \textbf{Technical Nature:}
          \begin{itemize}
              \item Uses web technologies and open standards (e.g., HTTPS, REST API).
          \end{itemize}
    \item \textbf{Data and File Formats:}
          \begin{itemize}
              \item Supports various file formats for conspectus uploads (PDF) and quiz question images (PNG, SVG, JPEG, WEBP, GIF).
          \end{itemize}
\end{itemize}

\subsection*{PostgreSQL Database Interface}
\begin{itemize}
    \item \textbf{Expected Volumes:}
          \begin{itemize}
              \item High volume of data transactions due to frequent read/write operations.
              \item Expected growth as more data is accumulated.
          \end{itemize}
    \item \textbf{Interaction Types:}
          \begin{itemize}
              \item Automated interactions from backend services.
              \item A few scheduled interactions.
          \end{itemize}
    \item \textbf{Interaction Automation Level:}
          \begin{itemize}
              \item All operations are automated.
          \end{itemize}
    \item \textbf{Transactional Nature:}
          \begin{itemize}
              \item Interactions are transactional to ensure data integrity.
          \end{itemize}
    \item \textbf{Criticality and Timeliness:}
          \begin{itemize}
              \item High criticality for data retrieval and storage operations.
              \item Timeliness crucial for real-time features like quizzes.
          \end{itemize}
    \item \textbf{Interaction Mode:}
          \begin{itemize}
              \item Entirely message-based.
          \end{itemize}
    \item \textbf{Security Requirements:}
          \begin{itemize}
              \item High security required, including data encryption.
          \end{itemize}
    \item \textbf{Service Levels:}
          \begin{itemize}
              \item High availability, scalability, and performance required.
          \end{itemize}
    \item \textbf{Technical Nature:}
          \begin{itemize}
              \item Standard SQL protocols.
          \end{itemize}
    \item \textbf{Data and File Formats:}
          \begin{itemize}
              \item Data stored in SQL relational tables.
          \end{itemize}
\end{itemize}

\subsection*{ChatGPT API}
\begin{itemize}
    \item \textbf{Expected Volumes:}
          \begin{itemize}
              \item Variable volumes based on user interaction levels with ChatGPT.
              \item Expected growth as more users utilize the feature.
          \end{itemize}
    \item \textbf{Interaction Types:}
          \begin{itemize}
              \item Ad hoc interactions initiated by user queries.
              \item Automated responses generated by the ChatGPT API.
          \end{itemize}
    \item \textbf{Interaction Automation Level:}
          \begin{itemize}
              \item All operations are automated.
          \end{itemize}
    \item \textbf{Transactional Nature:}
          \begin{itemize}
              \item Interactions are not strictly transactional but require timely processing.
          \end{itemize}
    \item \textbf{Criticality and Timeliness:}
          \begin{itemize}
              \item Timeliness is crucial for providing immediate responses to user queries.
          \end{itemize}
    \item \textbf{Interaction Mode:}
          \begin{itemize}
              \item Message-based, synchronous API calls.
          \end{itemize}
    \item \textbf{Security Requirements:}
          \begin{itemize}
              \item High level of security required, including secure API keys and encrypted communication.
          \end{itemize}
    \item \textbf{Service Levels:}
          \begin{itemize}
              \item Reliable and low-latency responses expected.
          \end{itemize}
    \item \textbf{Technical Nature:}
          \begin{itemize}
              \item REST API using JSON for data interchange.
          \end{itemize}
    \item \textbf{Data and File Formats:}
          \begin{itemize}
              \item Data exchanged in JSON format.
          \end{itemize}
\end{itemize}


\subsection{Other External Interdependencies}

No other external interdependencies are identified.


\subsection{Impact of the System on Its Environment}

\begin{itemize}
    \item Improved learning management for students and teachers.
    \item Potential reduction in the usage of other learning systems if EduPal is more efficient.
    \item Students may migrate their conspectuses and other content to EduPal.
\end{itemize}

(Not completely clear what should be written here. Is this like a traceability of things that should be done, or ideas for future?)
\subsection{Overall Completeness, Consistency, and Coherence}

\begin{itemize}
    \item Ensure all functionalities are well-integrated and user experiences are seamless.
    \item Maintain data consistency across all modules and interfaces.
    \item Plan for scalability and future integrations.
\end{itemize}

\subsection{Models}
\subsection*{Context Diagram}

\begin{figure}[H]
    \includedrawio[width=\textwidth]{viewpoints/context/context.drawio}
    \caption{Context diagram}
    \label{context-view-context-diagram}
\end{figure}

The context diagram visually represents EduPal as a central system interacting with its external entities.

\begin{itemize}
    \item \textbf{EduPal System:} Represents the main application.
    \item \textbf{Users (Students and Teachers):} Connected to EduPal with interactions for uploading and accessing course information, using other EduPal features.
    \item \textbf{PostgreSQL Database:} Connected to EduPal for data storage and retrieval.
\end{itemize}

\subsection{Perspectives}

\subsection*{The Security perspective}

\subsection*{The Performance and Scalability perspective}

\subsection*{The Availability and Resilience perspective}

\subsection*{The Evolution perspective}

\subsection*{The Accessibility perspective}

\subsection*{The Development Resource perspective}

\subsection*{The Internationalization perspective}

\subsection*{The Location perspective}

\subsection*{The Regulation perspective}

\subsection*{The Usability perspective}
