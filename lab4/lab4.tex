\documentclass[a4paper,12pt]{article}
\usepackage[top=2cm,bottom=2cm,left=3cm,right=1.5cm]{geometry}
\usepackage{hyperref}
\usepackage{graphicx}
\usepackage{float}
\hypersetup{
    colorlinks,
    linkcolor=black,
}
\tolerance 1414
\hbadness 1414
\emergencystretch 1.5em
\hfuzz 0.3pt
\widowpenalty=10000
\vfuzz \hfuzz

\newcommand{\includedrawio}[2][]{
    \immediate\write18{xvfb-run -a drawio --export #2 --output #2.pdf --crop --no-sandbox}
    \includegraphics[#1]{#2.pdf}
}

\begin{document}

\tableofcontents
\newpage

\section{Questions to ask}

\textbf{Context: what is missing?}

\textbf{Context viewpoint: what is missing?}

\textbf{Functional viewpoint: what is missing?}

\textbf{Information viewpoint: what is missing?}

Others context : system (database is in there), user. Text: system responsibilities. Description of external systems, what entities external system uses. Use cases from third lab.

Others functional: traceability between requirements and components.

Others Information: abstract than just database. (not tied to db or code), information access (who can access what and what are their permissions), information quality (what we store, what we don't store, how we filter data)

Concurrency view: system start, stop

Deployment view: deployment diagram, description, go through aspects that we must cover.

Perspectives after each viewpoint. Every perspective we selected must be covered, if its unrelated, then we write why its unrelated.

Ask about patterns and styles!

Ask if we should put OpenAI into context viewpoint.

Important!!! Reverse interface symbol in component diag.

\section{Deployment viewpoint}

\subsection{Runtime Platform Required}
\textbf{Concern:} Identifying the operating system or runtime environment needed to execute the software.

\textbf{Explanation:} The runtime platform for both the frontend and backend applications consists of Docker containers hosted on Azure Web App Services. The specific Docker images used are \texttt{node:20} for the frontend and \texttt{mcr.microsoft.com/dotnet/sdk:8.0} for the backend.

\subsection{Specification and Quantity of Hardware or Hosting Required}
\textbf{Concern:} Determining the necessary hardware resources and the number of physical or virtual machines.

\textbf{Explanation:}
\begin{itemize}
    \item \textbf{Frontend and Backend Azure Web App Services:}
    \begin{itemize}
        \item Memory: 16 GB each
        \item vCPU: 8 each
        \item Storage: 512 GB each
        \item Slots: 3 each (for deployment slots)
    \end{itemize}
    \item \textbf{Azure PostgreSQL Server:}
    \begin{itemize}
        \item vCPUs: 4
        \item Memory: 16 GB
        \item Storage: 512 GB
    \end{itemize}
    \item \textbf{Azure Container Registry} is used to store Docker images securely and manage them efficiently.
\end{itemize}
The deployment separates frontend and backend resources into different Azure Resource Groups, which improves manageability and resource allocation.

\subsection{Third-Party Software Requirements}
\textbf{Concern:} Identifying external software dependencies required by the system.

\textbf{Explanation:} The deployment relies on several Azure services:
\begin{itemize}
    \item \textbf{Azure Web App Services} for hosting the frontend and backend applications.
    \item \textbf{Azure PostgreSQL Server} for database management.
    \item \textbf{Azure Container Registry} for storing Docker images.
\end{itemize}
These third-party services are integral to the deployment process, ensuring that the applications are hosted, managed, and scaled effectively.

\subsection{Technology Compatibility}
\textbf{Concern:} Ensuring that all components of the system are compatible with each other.

\textbf{Explanation:} The chosen technologies are highly compatible:
\begin{itemize}
    \item \textbf{Docker} ensures consistency across different environments.
    \item \textbf{Node.js and .NET} SDKs are well-supported on Azure.
    \item \textbf{Azure Web App Services} are designed to work seamlessly with Docker containers and Azure databases.
    \item \textbf{GitLab} CI/CD integrates smoothly with Azure services, automating the deployment process.
\end{itemize}

\subsection{Network Requirements}
\textbf{Concern:} Defining the network infrastructure needed for the system.

\textbf{Explanation:}
\begin{itemize}
    \item \textbf{Private Endpoint} for the Azure PostgreSQL server ensures secure communication between the backend service and the database.
    \item \textbf{VNet Integration} provides a secure network boundary for the backend services, preventing exposure to the public internet.
    \item \textbf{Region:} North Europe, ensuring proximity to the target user base for reduced latency.
\end{itemize}

\subsection{Network Capacity Required}
\textbf{Concern:} Assessing the necessary network bandwidth and throughput.

\textbf{Explanation:} The manual scaling strategy implies that network capacity must be monitored and adjusted based on the load. As traffic increases, the number of instances for both frontend and backend services can be manually scaled to handle the additional load, ensuring that network performance remains optimal.

\subsection{Physical Constraints}
\textbf{Concern:} Identifying any physical limitations or environmental factors that could affect the deployment.

\textbf{Explanation:} The deployment is entirely cloud-based, hosted on Microsoft Azure's North Europe region. Physical constraints like data center location, hardware maintenance, and physical security are managed by Azure. The deployment takes advantage of Azure's global infrastructure, ensuring high availability and reliability without the need for on-premises hardware.

\section{Context viewpoint}

\begin{figure}[H]
    \includedrawio[width=\textwidth]{viewpoints/context/context.drawio}
    \caption{Context diagram}
    \label{context-view-context-diagram}
\end{figure}


\subsection{System Scope and Responsibilities}

\subsection*{Scope}
EduPal is a learning aid platform for students and teachers. It allows educators to upload course information, and students to access learning material and tools.

\begin{itemize}
    \item Real-Time Multiple Participant Quizzes: This will allow users to create quiz rooms, share access codes, and compete in real-time quizzes with other users, with results displayed immediately.
\end{itemize}


\subsection{Identity, Nature and Characteristics of External Entities and Services and Data Used}

\subsection*{External Entities}
\begin{itemize}
    \item \textbf{Users (Students and Teachers):} Interact with EduPal to upload, manage, and access course content or take quizzes, utilise the features such as Pomodoro timer, quizzes, etc.
    \item \textbf{Object Storage System:} The external data storage system for EduPal. Stores all conspectus files.
\end{itemize}

\subsection*{Users}
\begin{itemize}
    \item Require a reliable and user-friendly interface.
    \item Access the system from various devices and locations.
    \item Depend on system availability and performance for effective learning.
\end{itemize}

\subsection*{Object Storage System}
\begin{itemize}
    \item High availability and reliability.
    \item Capable of handling large amounts of data.
\end{itemize}

\subsection*{Data Used}
\begin{itemize}
    \item \textbf{User Data:} Includes quiz related data, such as results.
    \item \textbf{Quiz Data:} Includes questions, answers, and results of the real-time quizzes.
\end{itemize}


\subsection{Identity and Responsibilities, and Nature and Characteristics of External Interfaces}

\subsection*{External Interfaces}
\begin{itemize}
    \item \textbf{User Interface:}
          \begin{itemize}
              \item \textbf{Identity:} The web-based interface accessed by users (students and teachers) via browsers on various devices.
              \item \textbf{Responsibilities:}
                    Users do not have any responsibilities.
          \end{itemize}
    \item \textbf{Object Storage System Interface:}
          \begin{itemize}
              \item \textbf{Identity:} The interface through which the backend communicates with the object storage system.
              \item \textbf{Responsibilities:}
                    \begin{itemize}
                        \item Store and retrieve conspectus files.
                        \item Ensure high availability, reliability, and data integrity.
                        \item Scale to accommodate growing data volumes.
                    \end{itemize}
          \end{itemize}
\end{itemize}

\subsection*{User Interface}
\begin{itemize}
    \item \textbf{Expected Volumes:}
          \begin{itemize}
              \item High volume of requests during peak hours such as during exam periods or assignment deadlines.
              \item Varying data sizes, from small interactions (e.g., note-taking) to larger uploads (e.g., course material).
              \item Growth expected as user base increases.
          \end{itemize}
    \item \textbf{Interaction Types:}
          \begin{itemize}
              \item Mostly ad hoc interactions initiated by users.
              \item Automatic topic archival is scheduled.
          \end{itemize}
    \item \textbf{Interaction Automation Level:}
          \begin{itemize}
              \item Almost all interactions are fully manual, except some that are in between manual and automated, such as the Pomodoro timer.
          \end{itemize}
    \item \textbf{Transactional Nature:}
          \begin{itemize}
              \item None of the interactions are fully transactional, as all of them can be cancelled.
          \end{itemize}
    \item \textbf{Criticality and Timeliness:}
          \begin{itemize}
              \item Timely responses critical for a smooth user experience.
          \end{itemize}
    \item \textbf{Interaction Mode:}
          \begin{itemize}
              \item Primarily message-based, with occasional batch operations (e.g., bulk uploads).
          \end{itemize}
    \item \textbf{Security Requirements:}
          \begin{itemize}
              \item High level of security required, including user authentication and data confidentiality.
          \end{itemize}
    \item \textbf{Service Levels:}
          \begin{itemize}
              \item High availability and low latency expected to ensure user satisfaction.
          \end{itemize}
    \item \textbf{Technical Nature:}
          \begin{itemize}
              \item Uses web technologies and open standards (e.g., HTTPS, REST API).
          \end{itemize}
    \item \textbf{Data and File Formats:}
          \begin{itemize}
              \item Supports various file formats for conspectus uploads (PDF) and quiz question images (PNG, SVG, JPEG, WEBP, GIF).
          \end{itemize}
\end{itemize}

\subsection*{Object Storage System Interface}
\begin{itemize}
    \item \textbf{Expected Volumes:}
          \begin{itemize}
              \item Average data volume of data transactions dependent on frequent user interactions with topic conspectuses.
              \item Expected growth as more data is accumulated.
          \end{itemize}
    \item \textbf{Interaction Types:}
          \begin{itemize}
              \item Automated interactions from backend services.
              \item A few scheduled interactions.
          \end{itemize}
    \item \textbf{Interaction Automation Level:}
          \begin{itemize}
              \item All operations are automated.
          \end{itemize}
    \item \textbf{Transactional Nature:}
          \begin{itemize}
              \item Interactions are transactional to ensure data integrity.
          \end{itemize}
    \item \textbf{Criticality and Timeliness:}
          \begin{itemize}
              \item High criticality for data retrieval and storage operations.
              \item Timeliness is only somewhat important as users can endure some loading times.
          \end{itemize}
    \item \textbf{Interaction Mode:}
          \begin{itemize}
              \item Entirely message-based.
          \end{itemize}
    \item \textbf{Security Requirements:}
          \begin{itemize}
              \item High security required, including data encryption.
          \end{itemize}
    \item \textbf{Service Levels:}
          \begin{itemize}
              \item High availability, scalability, and performance required.
          \end{itemize}
    \item \textbf{Technical Nature:}
          \begin{itemize}
              \item Standard Object Storage System protocols.
          \end{itemize}
    \item \textbf{Data and File Formats:}
          \begin{itemize}
              \item Data is stored in large binary objects.
          \end{itemize}
\end{itemize}


\subsection{Other External Interdependencies}

No other external interdependencies are identified.


\subsection{Impact of the System on Its Environment}

\begin{itemize}
    \item Improved learning management for students and teachers.
    \item Potential reduction in the usage of other learning systems if EduPal is more efficient.
    \item Students may migrate their conspectuses and other content to EduPal.
\end{itemize}


\subsection{Overall Completeness, Consistency, and Coherence}

\begin{itemize}
    \item Ensure all functionalities are well-integrated and user experiences are seamless.
    \item Maintain data consistency across all modules and interfaces.
    \item Plan for scalability and future integrations.
\end{itemize}


\subsection{Perspectives}


\subsection*{The Security perspective}


\subsection*{The Performance and Scalability perspective}


\subsection*{The Availability and Resilience perspective}


\subsection*{The Evolution perspective}


\subsection*{The Location perspective}


\subsection*{The Regulation perspective}


\subsection*{The Usability perspective}

\section{Context viewpoint}

\begin{figure}[H]
    \includedrawio[width=\textwidth]{viewpoints/context/context.drawio}
    \caption{Context diagram}
    \label{context-view-context-diagram}
\end{figure}


\subsection{System Scope and Responsibilities}

\subsection*{Scope}
EduPal is a learning aid platform for students and teachers. It allows educators to upload course information, and students to access learning material and tools.

\begin{itemize}
    \item Real-Time Multiple Participant Quizzes: This will allow users to create quiz rooms, share access codes, and compete in real-time quizzes with other users, with results displayed immediately.
\end{itemize}


\subsection{Identity, Nature and Characteristics of External Entities and Services and Data Used}

\subsection*{External Entities}
\begin{itemize}
    \item \textbf{Users (Students and Teachers):} Interact with EduPal to upload, manage, and access course content or take quizzes, utilise the features such as Pomodoro timer, quizzes, etc.
    \item \textbf{Object Storage System:} The external data storage system for EduPal. Stores all conspectus files.
\end{itemize}

\subsection*{Users}
\begin{itemize}
    \item Require a reliable and user-friendly interface.
    \item Access the system from various devices and locations.
    \item Depend on system availability and performance for effective learning.
\end{itemize}

\subsection*{Object Storage System}
\begin{itemize}
    \item High availability and reliability.
    \item Capable of handling large amounts of data.
\end{itemize}

\subsection*{Data Used}
\begin{itemize}
    \item \textbf{User Data:} Includes quiz related data, such as results.
    \item \textbf{Quiz Data:} Includes questions, answers, and results of the real-time quizzes.
\end{itemize}


\subsection{Identity and Responsibilities, and Nature and Characteristics of External Interfaces}

\subsection*{External Interfaces}
\begin{itemize}
    \item \textbf{User Interface:}
          \begin{itemize}
              \item \textbf{Identity:} The web-based interface accessed by users (students and teachers) via browsers on various devices.
              \item \textbf{Responsibilities:}
                    Users do not have any responsibilities.
          \end{itemize}
    \item \textbf{Object Storage System Interface:}
          \begin{itemize}
              \item \textbf{Identity:} The interface through which the backend communicates with the object storage system.
              \item \textbf{Responsibilities:}
                    \begin{itemize}
                        \item Store and retrieve conspectus files.
                        \item Ensure high availability, reliability, and data integrity.
                        \item Scale to accommodate growing data volumes.
                    \end{itemize}
          \end{itemize}
\end{itemize}

\subsection*{User Interface}
\begin{itemize}
    \item \textbf{Expected Volumes:}
          \begin{itemize}
              \item High volume of requests during peak hours such as during exam periods or assignment deadlines.
              \item Varying data sizes, from small interactions (e.g., note-taking) to larger uploads (e.g., course material).
              \item Growth expected as user base increases.
          \end{itemize}
    \item \textbf{Interaction Types:}
          \begin{itemize}
              \item Mostly ad hoc interactions initiated by users.
              \item Automatic topic archival is scheduled.
          \end{itemize}
    \item \textbf{Interaction Automation Level:}
          \begin{itemize}
              \item Almost all interactions are fully manual, except some that are in between manual and automated, such as the Pomodoro timer.
          \end{itemize}
    \item \textbf{Transactional Nature:}
          \begin{itemize}
              \item None of the interactions are fully transactional, as all of them can be cancelled.
          \end{itemize}
    \item \textbf{Criticality and Timeliness:}
          \begin{itemize}
              \item Timely responses critical for a smooth user experience.
          \end{itemize}
    \item \textbf{Interaction Mode:}
          \begin{itemize}
              \item Primarily message-based, with occasional batch operations (e.g., bulk uploads).
          \end{itemize}
    \item \textbf{Security Requirements:}
          \begin{itemize}
              \item High level of security required, including user authentication and data confidentiality.
          \end{itemize}
    \item \textbf{Service Levels:}
          \begin{itemize}
              \item High availability and low latency expected to ensure user satisfaction.
          \end{itemize}
    \item \textbf{Technical Nature:}
          \begin{itemize}
              \item Uses web technologies and open standards (e.g., HTTPS, REST API).
          \end{itemize}
    \item \textbf{Data and File Formats:}
          \begin{itemize}
              \item Supports various file formats for conspectus uploads (PDF) and quiz question images (PNG, SVG, JPEG, WEBP, GIF).
          \end{itemize}
\end{itemize}

\subsection*{Object Storage System Interface}
\begin{itemize}
    \item \textbf{Expected Volumes:}
          \begin{itemize}
              \item Average data volume of data transactions dependent on frequent user interactions with topic conspectuses.
              \item Expected growth as more data is accumulated.
          \end{itemize}
    \item \textbf{Interaction Types:}
          \begin{itemize}
              \item Automated interactions from backend services.
              \item A few scheduled interactions.
          \end{itemize}
    \item \textbf{Interaction Automation Level:}
          \begin{itemize}
              \item All operations are automated.
          \end{itemize}
    \item \textbf{Transactional Nature:}
          \begin{itemize}
              \item Interactions are transactional to ensure data integrity.
          \end{itemize}
    \item \textbf{Criticality and Timeliness:}
          \begin{itemize}
              \item High criticality for data retrieval and storage operations.
              \item Timeliness is only somewhat important as users can endure some loading times.
          \end{itemize}
    \item \textbf{Interaction Mode:}
          \begin{itemize}
              \item Entirely message-based.
          \end{itemize}
    \item \textbf{Security Requirements:}
          \begin{itemize}
              \item High security required, including data encryption.
          \end{itemize}
    \item \textbf{Service Levels:}
          \begin{itemize}
              \item High availability, scalability, and performance required.
          \end{itemize}
    \item \textbf{Technical Nature:}
          \begin{itemize}
              \item Standard Object Storage System protocols.
          \end{itemize}
    \item \textbf{Data and File Formats:}
          \begin{itemize}
              \item Data is stored in large binary objects.
          \end{itemize}
\end{itemize}


\subsection{Other External Interdependencies}

No other external interdependencies are identified.


\subsection{Impact of the System on Its Environment}

\begin{itemize}
    \item Improved learning management for students and teachers.
    \item Potential reduction in the usage of other learning systems if EduPal is more efficient.
    \item Students may migrate their conspectuses and other content to EduPal.
\end{itemize}


\subsection{Overall Completeness, Consistency, and Coherence}

\begin{itemize}
    \item Ensure all functionalities are well-integrated and user experiences are seamless.
    \item Maintain data consistency across all modules and interfaces.
    \item Plan for scalability and future integrations.
\end{itemize}


\subsection{Perspectives}


\subsection*{The Security perspective}


\subsection*{The Performance and Scalability perspective}


\subsection*{The Availability and Resilience perspective}


\subsection*{The Evolution perspective}


\subsection*{The Location perspective}


\subsection*{The Regulation perspective}


\subsection*{The Usability perspective}

\section{Functional viewpoint}

\subsection{Components}

\begin{figure}[H]
  \includedrawio{viewpoints/functional/components.drawio}
  \label{fig:viewpoint-functional-components}
  \caption{Component diagram}
\end{figure}

Every interface without explicitly specified protocol is done as a code interface with instance classes knowing about each other via dependency injection.

\begin{description}
  \item[User Web Browser:] Component responsible for communication with front-facing EduPal interfaces as well as displaying EduPal web pages. Web browser, being external and user controlled entity, is not trusted and its requests are verified by front-facing interfaces.
  \item[React Web Page Server:] Component responsible for serving EduPal React webpages to the web browser. This component does not have access to other parts of the EduPal system and is only responsible for delivery of hypertext, media and code assets that are executed and rendered on the web browser.
  \item[API Service:] Component responsible for handling user requests, validating them and communicating with functionality related services (Quiz management, Topic management) to perform the operation and prepare the response.
  \item[Auhtorization Service:] Component responsible for determining whether certain type of request can be processed (checks whether user has the privilege) and whether user has the right to perform certain operations (such as quiz creation).
  \item[Topic and Quiz Management Services:] Components that API Service is communicating with. These components offer creation, update, delete and other data management commands and are responsible for checking if operation can be performed based on business rules.
  \item[Data Storage components:] Components responsible for abstracting data access and update logic. Quiz Game Session Data Storage is extracted as a separate component as a separation of concerns because quiz game sessions are short, require frequent updates and are unrelated to the rest of EduPal system.
  \item[Quiz Game Session Hub:] Component responsible for maintaining SignalR communication with the web browser, performing identity check via authorization service and passing requests to Quiz Game Session Handler as well as reacting to Quiz Game Session Handler events.
  \item[Quiz Game Session Handler:] Component responsible for handling the quiz game business logic, grading answers and communicating with the data storage.

\end{description}

\section{Information viewpoint}

\subsection{Information Structure and Content}

\begin{figure}[H]
  \includedrawio[width=\textwidth]{viewpoints/information/entities.drawio}
  \caption{Persistent entity diagram}
  \label{fig:viewpoint-information-entity}
\end{figure}

\subsection{Information Purpose and Usage}

All presented entities are required for EduPal operation with quiz functionality. User, Topic, Quiz, Question and Option entities are responsible for static data storage, while Quiz Game Session, Quiz Game Player and Quiz Game Player Answer are used for multiplayer quizzes. Entities are used only for OLTP processing as no data analysis requirements exist. Purposes and caveats of certain entities and relations:

\begin{enumerate}
  \item Topic may have the owner and quiz has the topic - required to trace who is the owner of a particular quiz.
  \item Quiz Game Session stores pointers to shuffled questions (it is expected that implementation will be using id's of questions) to ensure that order of questions in every game is different.
  \item Quiz Game Player is a separate entity of user to decrease coupling between entities and ensure that possible future requirement of anonymous players will be possible to implement without complex changes.
  \item Question may point to an image URI stored in external object storage system.
  \item Quiz Game session is only stored while session is active.
  \item Last access field is need for archivation purposes in the future - entities that have not been used for a long time will move to future archive database.
\end{enumerate}

\subsection{Information Ownership}

EduPal uses a single centric database to simplify operations. This ensures that we do not have to deal with information ownership issues.

\subsection{Identifiers}

Every entity must have an explicit Id attribute with unique (for the entity group) UUID value. Id attribute acts as a primary key and cannot change throughout the operation. The decision to have explicit primary key attribute and not to use some other data (such as username) as a primary key ensures that every entity has a similar structure and constraints of other attributes can be changed (username may or may not be unique). Every image has a unique URI and it is the responsibility of external object storage system to ensure this.

\subsection{Volatility of Information Semantics}

Database schema and its migrations are tracked by Entity Framework Core ORM. Since the database is only accessed by the monolithic EduPal application and none of the other systems, this management option is enough and allows to save developer time and cut down unnecessary costs.

\subsection{Information Storage Models}

Static data will be stored in a relational database - this ensures data consistency and allows to specify proper constraints for the data. Images will be stored in external object storage system (that is optimized for quick access and upload of 1-100Mb files).

Quiz game sessions require fast data reads and writes and are deleted upon completion. To achieve high performance NoSQL document or key-value database with efficient caching capabilities should be used. A support for in-memory operation would also be beneficial.

\subsection{Information Flow}

Information is created, accessed, destroyed and modified by the business layer of the EduPal.

\subsection{Information consistency}

In order to increase performance during request processing EduPal chooses to use eventual consistency model - data access layer may cache certain data changes and not update the database in case the entity is commonly queried or updated.

\subsection{Archiving and Information Retention}

EduPal is still a young and developing system without a huge amount of data. Because of that and not a huge amount of information stored (Every question is typically one sentence, option is a few words), we currently do not need any automatic or even manual archivation processes. In the future, the data access layer will need to be extended - we will add archive database that old entities will go to. Data access layer and business layer will need to be updated to move information between databases.

\subsection{Security perspective}

Every static information is publicly available to the end users, therefore the only resources we need to securely store are user credentials. Because of low amount of entities with high sensitivity, we do not think it is necessary to separate data into a separate database. Data integrity is ensured by the fact that database is only accessed and updated by the EduPal system. Since every entity has a single owner, every data state change can be traced to the user.

\subsection{Performance and Scalability}

In order to increase performance, quiz sessions are stored in a NoSQL database. Scalability is primarily limited by a monolithic system architecture and for such architecture single database is enough.

\subsection{Availability and recovery}

In case of a controllable system crash (such as unhandled exception or similar), before the shutdown the system data access layer will perform the synchronization with the database. In case of a serious and uncontrollable crash (hardware failure or recovery crash and similar) temporary data will be lost. While this is a critical issue, considering the scope of EduPal the scenario is acceptable. Additionally, to minimize the losses, data access layer must synchronize the changes that are at least one hour old.

\subsection{Evolution perspective}

Information model is flexible as entities are related only when necessary and only required attributes are stored ensuring that the information model may evolve as need. Additionally, Some entities (such as Quiz Game Player and User) are decoupled to minimize future changes in case of new requirements.

\subsection{Location perspective}

Unrelated, since data is not distributed.

\subsection{Regulation perspective}

Only necessary data that is identified by terms of service is collected. EduPal does not archive data of deleted users and instead completely deletes the data entries (in case of user deletion it means that their topics and quizzes will be deleted). Data that is necessary only during quiz session is deleted by the end of it.

\subsection{Usability perspective}

High information quality and properly defined relations make the information highly usable and convenient to operate in business layers.


\listoffigures

\end{document}
