\documentclass[a4paper,12pt]{article}
\usepackage[top=2cm,bottom=2cm,left=3cm,right=1.5cm]{geometry}
\usepackage{hyperref}
\usepackage{graphicx}
\usepackage{float}
\hypersetup{
    colorlinks,
    linkcolor=black,
}
\tolerance 1414
\hbadness 1414
\emergencystretch 1.5em
\hfuzz 0.3pt
\widowpenalty=10000
\vfuzz \hfuzz

\newcommand{\includedrawio}[2][]{
    \immediate\write18{xvfb-run -a drawio --export #2 --output #2.pdf --crop --no-sandbox}
    \includegraphics[#1]{#2.pdf}
}

\begin{document}

\tableofcontents
\newpage

\section{Questions to ask}

\textbf{Context: what is missing?}

\textbf{Context viewpoint: what is missing?}

\textbf{Functional viewpoint: what is missing?}

\textbf{Information viewpoint: what is missing?}

Others context : system (database is in there), user. Text: system responsibilities. Description of external systems, what entities external system uses. Use cases from third lab.

Others functional: traceability between requirements and components.

Others Information: abstract than just database. (not tied to db or code), information access (who can access what and what are their permissions), information quality (what we store, what we don't store, how we filter data)

Concurrency view: system start, stop

Deployment view: deployment diagram, description, go through aspects that we must cover.

Perspectives after each viewpoint. Every perspective we selected must be covered, if its unrelated, then we write why its unrelated.

Ask about patterns and styles!

Ask if we should put OpenAI into context viewpoint.

Important!!! Reverse interface symbol in component diag.

\section{Deployment viewpoint}

\subsection{Runtime Platform Required}
\textbf{Concern:} Identifying the operating system or runtime environment needed to execute the software.

\textbf{Explanation:} The runtime platform for both the frontend and backend applications consists of Docker containers hosted on Azure Web App Services. The specific Docker images used are \texttt{node:20} for the frontend and \texttt{mcr.microsoft.com/dotnet/sdk:8.0} for the backend.

\subsection{Specification and Quantity of Hardware or Hosting Required}
\textbf{Concern:} Determining the necessary hardware resources and the number of physical or virtual machines.

\textbf{Explanation:}
\begin{itemize}
    \item \textbf{Frontend and Backend Azure Web App Services:}
    \begin{itemize}
        \item Memory: 16 GB each
        \item vCPU: 8 each
        \item Storage: 512 GB each
        \item Slots: 3 each (for deployment slots)
    \end{itemize}
    \item \textbf{Azure PostgreSQL Server:}
    \begin{itemize}
        \item vCPUs: 4
        \item Memory: 16 GB
        \item Storage: 512 GB
    \end{itemize}
    \item \textbf{Azure Container Registry} is used to store Docker images securely and manage them efficiently.
\end{itemize}
The deployment separates frontend and backend resources into different Azure Resource Groups, which improves manageability and resource allocation.

\subsection{Third-Party Software Requirements}
\textbf{Concern:} Identifying external software dependencies required by the system.

\textbf{Explanation:} The deployment relies on several Azure services:
\begin{itemize}
    \item \textbf{Azure Web App Services} for hosting the frontend and backend applications.
    \item \textbf{Azure PostgreSQL Server} for database management.
    \item \textbf{Azure Container Registry} for storing Docker images.
\end{itemize}
These third-party services are integral to the deployment process, ensuring that the applications are hosted, managed, and scaled effectively.

\subsection{Technology Compatibility}
\textbf{Concern:} Ensuring that all components of the system are compatible with each other.

\textbf{Explanation:} The chosen technologies are highly compatible:
\begin{itemize}
    \item \textbf{Docker} ensures consistency across different environments.
    \item \textbf{Node.js and .NET} SDKs are well-supported on Azure.
    \item \textbf{Azure Web App Services} are designed to work seamlessly with Docker containers and Azure databases.
    \item \textbf{GitLab} CI/CD integrates smoothly with Azure services, automating the deployment process.
\end{itemize}

\subsection{Network Requirements}
\textbf{Concern:} Defining the network infrastructure needed for the system.

\textbf{Explanation:}
\begin{itemize}
    \item \textbf{Private Endpoint} for the Azure PostgreSQL server ensures secure communication between the backend service and the database.
    \item \textbf{VNet Integration} provides a secure network boundary for the backend services, preventing exposure to the public internet.
    \item \textbf{Region:} North Europe, ensuring proximity to the target user base for reduced latency.
\end{itemize}

\subsection{Network Capacity Required}
\textbf{Concern:} Assessing the necessary network bandwidth and throughput.

\textbf{Explanation:} The manual scaling strategy implies that network capacity must be monitored and adjusted based on the load. As traffic increases, the number of instances for both frontend and backend services can be manually scaled to handle the additional load, ensuring that network performance remains optimal.

\subsection{Physical Constraints}
\textbf{Concern:} Identifying any physical limitations or environmental factors that could affect the deployment.

\textbf{Explanation:} The deployment is entirely cloud-based, hosted on Microsoft Azure's North Europe region. Physical constraints like data center location, hardware maintenance, and physical security are managed by Azure. The deployment takes advantage of Azure's global infrastructure, ensuring high availability and reliability without the need for on-premises hardware.

\section{Context}

EduPal is a learning aid business which helps both students and teachers by allowing educators to conveniently upload course information for students to learn. EduPal offers unique features such as pomodoro timer, note creation and export in pdf, setting goals, chatGPT integration. EduPal lets users have subjects, create topics inside subjects and upload conspectuses.

\subsection{Tech stack overview}

EduPal follows monolithic architecture approach - it consists of ASP.NET 8 API backend and React frontend that communicates with backend via REST API.

\subsection{System change overview}

EduPal is bound to grow and evolve and the latest addition we are considering is a multiple participant real-time quizzing. This feature will allow users to create a quiz room, receive a randomly generated code, share it with other users, and compete in real-time to see who can answer the most questions correctly. Participants will be answering the same questions at the same time. The system will display the results of the quiz to all participants and show their place among other users after each question.

\subsection{System change requirements}

\subsubsection{Functional requirements}

\newlist{frlist}{enumerate}{1}
\setlist[frlist]{label=\textbf{FR\arabic*.}, align=left, leftmargin=*}

\begin{frlist}
    \item System shall allow users to list quizzes.
    \item System shall ensure that only logged in users can access quizzes.
    \item System shall only allow creation, editing and deletion of the topic quizzes to owners of that topic.
    \item System shall allow topic owner to delete any quiz that belongs to their topic.
    \item System shall allow topic owner to create the quiz in their topic.
    \item System shall allow topic owner to edit any quiz in their topic.
    \item System shall prompt for confirmation when deleting quizzes.
    \item System shall support uploading image for questions.
    \item System shall support setting quiz name.
    \item System shall support adding, editing and deleting quiz questions and options.
    \item System shall support multiple participant quizzes.
    \item System shall allow users to create a quiz room.
    \item System shall generate a random code for the quiz room.
    \item System shall allow the users to join a quiz room by entering the code.
    \item System shall display feedback and their current position to all participants after answering a question.
    \item System shall allow a participant to leave a quiz at any time.
    \item System shall ensure that there are no active quiz rooms with the same room code.
\end{frlist}


\subsubsection{Non-functional requirements}

\newlist{nfrlist}{enumerate}{1}
\setlist[nfrlist]{label=\textbf{NFR\arabic*.}, align=left, leftmargin=*}

\begin{nfrlist}
    \item System shall support PNG, SVG, JPEG, WEBP, GIF question image formats.
    \item System shall support at least 100 participants in a quiz room.
    \item System shall support at least 200 quiz rooms at the same time.
\end{nfrlist}

\subsection{Covered perspectives}

\begin{enumerate}
    \item Security perspective
    \item Performance and Scalability perspective
    \item Availability and Resilience perspective
    \item Evolution perspective
    \item Location perspective
    \item Regulation perspective
    \item Usability perspective
\end{enumerate}

Development resource perspective is not covered since some of the considerations were outlined in previous business analysis documents and quizzes are not a new system but rather a functional change in existing system that was developed and maintained by our team.

Accessibility perspective is not covered since our funds and development time is limited and it is not the primary concern of current stakeholders. Additionally, many accessibility features do not require a drastic change in any of the viewpoints and some of the accessibility concerns are covered by usability perspective.

Internationalization perspective is not covered since the target users are english speaking and stakeholders do not want to invest money into internationalization until a feature is considered a success.



\section{Context}

EduPal is a learning aid business which helps both students and teachers by allowing educators to conveniently upload course information for students to learn. EduPal offers unique features such as pomodoro timer, note creation and export in pdf, setting goals, chatGPT integration. EduPal lets users have subjects, create topics inside subjects and upload conspectuses.

\subsection{Tech stack overview}

EduPal follows monolithic architecture approach - it consists of ASP.NET 8 API backend and React frontend that communicates with backend via REST API.

\subsection{System change overview}

EduPal is bound to grow and evolve and the latest addition we are considering is a multiple participant real-time quizzing. This feature will allow users to create a quiz room, receive a randomly generated code, share it with other users, and compete in real-time to see who can answer the most questions correctly. Participants will be answering the same questions at the same time. The system will display the results of the quiz to all participants and show their place among other users after each question.

\subsection{System change requirements}

\subsubsection{Functional requirements}

\newlist{frlist}{enumerate}{1}
\setlist[frlist]{label=\textbf{FR\arabic*.}, align=left, leftmargin=*}

\begin{frlist}
    \item System shall allow users to list quizzes.
    \item System shall ensure that only logged in users can access quizzes.
    \item System shall only allow creation, editing and deletion of the topic quizzes to owners of that topic.
    \item System shall allow topic owner to delete any quiz that belongs to their topic.
    \item System shall allow topic owner to create the quiz in their topic.
    \item System shall allow topic owner to edit any quiz in their topic.
    \item System shall prompt for confirmation when deleting quizzes.
    \item System shall support uploading image for questions.
    \item System shall support setting quiz name.
    \item System shall support adding, editing and deleting quiz questions and options.
    \item System shall support multiple participant quizzes.
    \item System shall allow users to create a quiz room.
    \item System shall generate a random code for the quiz room.
    \item System shall allow the users to join a quiz room by entering the code.
    \item System shall display feedback and their current position to all participants after answering a question.
    \item System shall allow a participant to leave a quiz at any time.
    \item System shall ensure that there are no active quiz rooms with the same room code.
\end{frlist}


\subsubsection{Non-functional requirements}

\newlist{nfrlist}{enumerate}{1}
\setlist[nfrlist]{label=\textbf{NFR\arabic*.}, align=left, leftmargin=*}

\begin{nfrlist}
    \item System shall support PNG, SVG, JPEG, WEBP, GIF question image formats.
    \item System shall support at least 100 participants in a quiz room.
    \item System shall support at least 200 quiz rooms at the same time.
\end{nfrlist}

\subsection{Covered perspectives}

\begin{enumerate}
    \item Security perspective
    \item Performance and Scalability perspective
    \item Availability and Resilience perspective
    \item Evolution perspective
    \item Location perspective
    \item Regulation perspective
    \item Usability perspective
\end{enumerate}

Development resource perspective is not covered since some of the considerations were outlined in previous business analysis documents and quizzes are not a new system but rather a functional change in existing system that was developed and maintained by our team.

Accessibility perspective is not covered since our funds and development time is limited and it is not the primary concern of current stakeholders. Additionally, many accessibility features do not require a drastic change in any of the viewpoints and some of the accessibility concerns are covered by usability perspective.

Internationalization perspective is not covered since the target users are english speaking and stakeholders do not want to invest money into internationalization until a feature is considered a success.



\section{Functional viewpoint}

\subsection{Internal structure}

EduPal follows monolithic service oriented architecture. Each service (also called component in the document) has a defined in-code interface that other services interact with. This allows system to be adequately decoupled while maintaining consistency. Services are tiered into 3 categories:

\begin{enumerate}
  \item Data access services, services that store the data themselves communicate with external system or database to store and retrieve the data. Such service does not perform any business logic related processing. Its interfaces are used by business logic layer.
  \item Business logic services, services that perform the processing, determines if operation is valid, if needed, communicates with the data access layer services.
  \item Interface services, services that are responsible for communicating with external systems, converting requests into format that is acceptable by a particular business logic service, and turning business logic service responses into a valid communication object between the interfaces.
\end{enumerate}

\subsection{Components}

\begin{figure}[H]
  \includedrawio[width=\textwidth]{viewpoints/functional/components.drawio}
  \label{fig:viewpoint-functional-components}
  \caption{Component diagram}
\end{figure}

Every interface without explicitly specified protocol is done as a code interface with instance classes knowing about each other via dependency injection.

\begin{description}
  \item[User Web Browser:] Component responsible for communication with front-facing EduPal interfaces as well as displaying EduPal web pages. Web browser, being external and user controlled entity, is not trusted and its requests are verified by front-facing interfaces.
  \item[React Web Page Server:] Component responsible for serving EduPal React webpages to the web browser. This component does not have access to other parts of the EduPal system and is only responsible for delivery of hypertext, media and code assets that are executed and rendered on the web browser.
  \item[API Service:] Component responsible for handling user requests, validating them and communicating with functionality related services (Quiz management, Topic management) to perform the operation and prepare the response.
  \item[Auhtorization Service:] Component responsible for determining whether certain type of request can be processed (checks whether user has the privilege) and whether user has the right to perform certain operations (such as quiz creation).
  \item[Topic and Quiz Management Services:] Components that API Service is communicating with. These components offer creation, update, delete and other data management commands and are responsible for checking if operation can be performed based on business rules.
  \item[Object Data Storage Adapter:] Component responsible for communication with the data object storage system in order to store or retrieve the data. 
  \item[Data Storage components:] Components responsible for abstracting data access and update logic. Quiz Game Session Data Storage is extracted as a separate component as a separation of concerns because quiz game sessions are short, require frequent updates and are unrelated to the rest of EduPal system.
  \item[Quiz Game Session Hub:] Component responsible for maintaining SignalR communication with the web browser, performing identity check via authorization service and passing requests to Quiz Game Session Handler as well as reacting to Quiz Game Session Handler events.
  \item[Quiz Game Session Handler:] Component responsible for handling the quiz game business logic, grading answers and communicating with the data storage.
\end{description}
  
\subsection{Characteristics}

\begin{description}
  \item[Separation of converns] Service responsibilities are based on their layer, that together form a whole system. Every layer has its own defined responsibilities.
  \item[Cohesive] Functionality related to the same entities (e.g. quiz operations) belongs to the same service component resulting in a highly cohesive system.
  \item[Consistency] Same design decision are applied through entirety of EduPal.
  \item[Coupling] Design results in a low coupling. Business logic does not depend on data access method and external communication method, API layer does not depend on data access method.
  \item[Extensibility and functional flexibility] System is extensible since it is enough to simply add new services to a particular layer. Functionality changes will require changing a particular set service or services, making the system flexible.
  \item[Simplicity] Layered structure is easy to understand, build and extend.
\end{description}
  
\subsection{Traceability}

\begin{figure}[H]
  \label{tab:functional-viewpoint-component-traceability}
  \centering
  \begin{adjustbox}{max width=\textwidth}
    \begin{tabular}{|l|l|l|l|l|l|l|l|l|l|}
      \hline
      \textbf{Component/Requirement} & \textbf{FR1} & \textbf{FR2} & \textbf{FR3} & \textbf{FR4-6,8-10} & \textbf{FR7} & \textbf{FR11-17}\\ \hline
      \textbf{React Web Page Server} & X & X & X & X & X & X \\ \hline
      \textbf{API Service} & X & X & X & X & ~ & X \\ \hline
      \textbf{Quiz Game Session Hub} & ~ & ~ & ~ & ~ & ~ & X \\ \hline
      \textbf{Authorization Service} & ~ & X & X & X & ~ & X \\ \hline
      \textbf{Quiz Game Session Handler} & ~ & ~ & ~ & ~ & ~ & X \\ \hline
      \textbf{Topic Management Service} & ~ & ~ & X & X & ~ & ~ \\ \hline
      \textbf{Quiz Management Service} & X & ~ & X & X & ~ & X \\ \hline
      \textbf{Data Storage} & X & ~ & X & X & ~ & X \\ \hline
      \textbf{Quiz Game Session Data Storage} & ~ & ~ & ~ & ~ & ~ & X \\ \hline
      \textbf{Object Data Storage Adapter} & ~ & ~ & ~ & X & ~ & X \\ \hline
  \end{tabular}
\end{adjustbox}
  \caption{Traceability between components and requirements}
\end{figure}

\subsection{Security Perspective}

\subsubsection{Resources}

System resource storage is responsibility of the data access layer, however it is not concern of the layer to determine who has the right to access the data and who does not. It is the business layer services that determine if resource can be accessed and the business layer depends on security context information (security cookies, auth tokens and similar) provided by interface layer. Authorization component is responsible for determining whether operation can be performed by the current user.

\subsubsection{Principle auth}

Interface layer is responsible for secure exchange of authentication tokens that Authorization component uses to determine, who is the user that performs the action and what privileges they have depending on the operation.

\subsubsection{Policies}

Authorization component will deny requests as unauthorized in the following cases:

\begin{enumerate}
  \item Any operation by unauthenticated user, except login and register.
  \item Quiz modification operations performed by the user, who is not owner of the topic.
\end{enumerate}

\subsubsection{Threats}

Since EduPal does not have high security requirements there are less possible threats. Attacker may take advantage of of lack of security processing of certain operations (e.g. developer's mistake) to do unauthorized action.

\subsubsection{Availability}

Interface layer is responsible for rate limiting of the connections.

\subsection{Performance perspective}

\subsubsection{Response time, throughput and scalability}

Data access layer may use caching to speed up database lookup. Services being stateless ensures that multiple threads can process multiple requests. However, a monolithic architecture heavily limits the throughput and does not scale well, since it is impossible to scale horizontally (and vertical scaling has an upper bound). Current architecture is acceptable since non-functional requirements are not demanding and other concerns (such as time to implement new features and data consistency) are more important.

\subsection{Availability and resilience}

No functional changes are required since there are no system availability requirements (such as offline mode).

\subsection{Evolution perspective}

Layered model ensures that it is easy to add new functionality or change existing one.

\subsection{Location perspective}

Unrelated.

\subsection{Regulation perspective}

Due to GDPR and other law enforcements, EduPal must ensure that business logic services only retrieve and store the data that is absolutely required for operation.

\subsection{Usability perspective}

API services must expose convenient API that does not require to do multiple requests. For example, it is considered inconvenient if adding 2 questions to a quiz requires 2 API calls, instead system must quiz edit API must be flexible and allow complex state changes within single request. Overall component structure does not have a great effect on usability and it is mostly the concern of interface layer.



\section{Information viewpoint}

\subsection{Persistent entities}

These are the entities stored by the Data Storage related to quiz functionality.

\begin{figure}[H]
  \includedrawio{viewpoints/information/entities.drawio}
  \caption{Persistent entity diagram}
  \label{fig:viewpoint-information-entity}
\end{figure}

\subsubsection{Storage}

Single PostgreSQL database is used to store persistent entities. This simplifies architecture, shortens development time and ensures that information is always in correct state. EntityFramework is used as an ORM. Code exposes Entity repositories as a way to abstract the ORM and allow to change the underlying storage backend.

\begin{figure}[H]
  \includedrawio[width=\textwidth]{viewpoints/information/data-access.drawio}
  \caption{Repository code diagram}
  \label{fig:viewpoint-information-repo}
\end{figure}

\subsection{Quiz game session entities}

These are the entities stored by the Quiz Game Session Data Storage component.

\begin{figure}[H]
  \includedrawio{viewpoints/information/quiz-game-entities.drawio}
  \caption{Quiz game session entity diagram}
  \label{fig:viewpoint-information-quiz-game-entities}
\end{figure}


\listoffigures

\end{document}
