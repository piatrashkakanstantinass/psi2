\documentclass[a4paper,12pt]{article}
\usepackage[top=2cm,bottom=2cm,left=3cm,right=1.5cm]{geometry}
\usepackage{hyperref}
\usepackage{graphicx}
\usepackage{float}
\hypersetup{
    colorlinks,
    linkcolor=black,
}
\tolerance 1414
\hbadness 1414
\emergencystretch 1.5em
\hfuzz 0.3pt
\widowpenalty=10000
\vfuzz \hfuzz

\newcommand{\includedrawio}[2][]{
    \immediate\write18{xvfb-run -a drawio --export #2 --output #2.pdf --crop --no-sandbox}
    \includegraphics[#1]{#2.pdf}
}

\begin{document}

\tableofcontents
\newpage

\section{Questions to ask}

\textbf{Context: what is missing?}

\textbf{Context viewpoint: what is missing?}

\textbf{Functional viewpoint: what is missing?}

\textbf{Information viewpoint: what is missing?}

Others context : system (database is in there), user. Text: system responsibilities. Description of external systems, what entities external system uses. Use cases from third lab.

Others functional: traceability between requirements and components.

Others Information: abstract than just database. (not tied to db or code), information access (who can access what and what are their permissions), information quality (what we store, what we don't store, how we filter data)

Concurrency view: system start, stop

Deployment view: deployment diagram, description, go through aspects that we must cover.

Perspectives after each viewpoint. Every perspective we selected must be covered, if its unrelated, then we write why its unrelated.

Ask about patterns and styles!

Ask if we should put OpenAI into context viewpoint.

Important!!! Reverse interface symbol in component diag.

\section{Context viewpoint}

\begin{figure}[H]
    \includedrawio[width=\textwidth]{viewpoints/context/context.drawio}
    \caption{Context diagram}
    \label{context-view-context-diagram}
\end{figure}


\subsection{System Scope and Responsibilities}

\subsection*{Scope}
EduPal is a learning aid platform for students and teachers. It allows educators to upload course information, and students to access learning material and tools.

\begin{itemize}
    \item Real-Time Multiple Participant Quizzes: This will allow users to create quiz rooms, share access codes, and compete in real-time quizzes with other users, with results displayed immediately.
\end{itemize}


\subsection{Identity, Nature and Characteristics of External Entities and Services and Data Used}

\subsection*{External Entities}
\begin{itemize}
    \item \textbf{Users (Students and Teachers):} Interact with EduPal to upload, manage, and access course content or take quizzes, utilise the features such as Pomodoro timer, quizzes, etc.
    \item \textbf{Object Storage System:} The external data storage system for EduPal. Stores all conspectus files.
\end{itemize}

\subsection*{Users}
\begin{itemize}
    \item Require a reliable and user-friendly interface.
    \item Access the system from various devices and locations.
    \item Depend on system availability and performance for effective learning.
\end{itemize}

\subsection*{Object Storage System}
\begin{itemize}
    \item High availability and reliability.
    \item Capable of handling large amounts of data.
\end{itemize}

\subsection*{Data Used}
\begin{itemize}
    \item \textbf{User Data:} Includes quiz related data, such as results.
    \item \textbf{Quiz Data:} Includes questions, answers, and results of the real-time quizzes.
\end{itemize}


\subsection{Identity and Responsibilities, and Nature and Characteristics of External Interfaces}

\subsection*{External Interfaces}
\begin{itemize}
    \item \textbf{User Interface:}
          \begin{itemize}
              \item \textbf{Identity:} The web-based interface accessed by users (students and teachers) via browsers on various devices.
              \item \textbf{Responsibilities:}
                    Users do not have any responsibilities.
          \end{itemize}
    \item \textbf{Object Storage System Interface:}
          \begin{itemize}
              \item \textbf{Identity:} The interface through which the backend communicates with the object storage system.
              \item \textbf{Responsibilities:}
                    \begin{itemize}
                        \item Store and retrieve conspectus files.
                        \item Ensure high availability, reliability, and data integrity.
                        \item Scale to accommodate growing data volumes.
                    \end{itemize}
          \end{itemize}
\end{itemize}

\subsection*{User Interface}
\begin{itemize}
    \item \textbf{Expected Volumes:}
          \begin{itemize}
              \item High volume of requests during peak hours such as during exam periods or assignment deadlines.
              \item Varying data sizes, from small interactions (e.g., note-taking) to larger uploads (e.g., course material).
              \item Growth expected as user base increases.
          \end{itemize}
    \item \textbf{Interaction Types:}
          \begin{itemize}
              \item Mostly ad hoc interactions initiated by users.
              \item Automatic topic archival is scheduled.
          \end{itemize}
    \item \textbf{Interaction Automation Level:}
          \begin{itemize}
              \item Almost all interactions are fully manual, except some that are in between manual and automated, such as the Pomodoro timer.
          \end{itemize}
    \item \textbf{Transactional Nature:}
          \begin{itemize}
              \item None of the interactions are fully transactional, as all of them can be cancelled.
          \end{itemize}
    \item \textbf{Criticality and Timeliness:}
          \begin{itemize}
              \item Timely responses critical for a smooth user experience.
          \end{itemize}
    \item \textbf{Interaction Mode:}
          \begin{itemize}
              \item Primarily message-based, with occasional batch operations (e.g., bulk uploads).
          \end{itemize}
    \item \textbf{Security Requirements:}
          \begin{itemize}
              \item High level of security required, including user authentication and data confidentiality.
          \end{itemize}
    \item \textbf{Service Levels:}
          \begin{itemize}
              \item High availability and low latency expected to ensure user satisfaction.
          \end{itemize}
    \item \textbf{Technical Nature:}
          \begin{itemize}
              \item Uses web technologies and open standards (e.g., HTTPS, REST API).
          \end{itemize}
    \item \textbf{Data and File Formats:}
          \begin{itemize}
              \item Supports various file formats for conspectus uploads (PDF) and quiz question images (PNG, SVG, JPEG, WEBP, GIF).
          \end{itemize}
\end{itemize}

\subsection*{Object Storage System Interface}
\begin{itemize}
    \item \textbf{Expected Volumes:}
          \begin{itemize}
              \item Average data volume of data transactions dependent on frequent user interactions with topic conspectuses.
              \item Expected growth as more data is accumulated.
          \end{itemize}
    \item \textbf{Interaction Types:}
          \begin{itemize}
              \item Automated interactions from backend services.
              \item A few scheduled interactions.
          \end{itemize}
    \item \textbf{Interaction Automation Level:}
          \begin{itemize}
              \item All operations are automated.
          \end{itemize}
    \item \textbf{Transactional Nature:}
          \begin{itemize}
              \item Interactions are transactional to ensure data integrity.
          \end{itemize}
    \item \textbf{Criticality and Timeliness:}
          \begin{itemize}
              \item High criticality for data retrieval and storage operations.
              \item Timeliness is only somewhat important as users can endure some loading times.
          \end{itemize}
    \item \textbf{Interaction Mode:}
          \begin{itemize}
              \item Entirely message-based.
          \end{itemize}
    \item \textbf{Security Requirements:}
          \begin{itemize}
              \item High security required, including data encryption.
          \end{itemize}
    \item \textbf{Service Levels:}
          \begin{itemize}
              \item High availability, scalability, and performance required.
          \end{itemize}
    \item \textbf{Technical Nature:}
          \begin{itemize}
              \item Standard Object Storage System protocols.
          \end{itemize}
    \item \textbf{Data and File Formats:}
          \begin{itemize}
              \item Data is stored in large binary objects.
          \end{itemize}
\end{itemize}


\subsection{Other External Interdependencies}

No other external interdependencies are identified.


\subsection{Impact of the System on Its Environment}

\begin{itemize}
    \item Improved learning management for students and teachers.
    \item Potential reduction in the usage of other learning systems if EduPal is more efficient.
    \item Students may migrate their conspectuses and other content to EduPal.
\end{itemize}


\subsection{Overall Completeness, Consistency, and Coherence}

\begin{itemize}
    \item Ensure all functionalities are well-integrated and user experiences are seamless.
    \item Maintain data consistency across all modules and interfaces.
    \item Plan for scalability and future integrations.
\end{itemize}


\subsection{Perspectives}


\subsection*{The Security perspective}


\subsection*{The Performance and Scalability perspective}


\subsection*{The Availability and Resilience perspective}


\subsection*{The Evolution perspective}


\subsection*{The Location perspective}


\subsection*{The Regulation perspective}


\subsection*{The Usability perspective}

\section{Context viewpoint}

\begin{figure}[H]
    \includedrawio[width=\textwidth]{viewpoints/context/context.drawio}
    \caption{Context diagram}
    \label{context-view-context-diagram}
\end{figure}


\subsection{System Scope and Responsibilities}

\subsection*{Scope}
EduPal is a learning aid platform for students and teachers. It allows educators to upload course information, and students to access learning material and tools.

\begin{itemize}
    \item Real-Time Multiple Participant Quizzes: This will allow users to create quiz rooms, share access codes, and compete in real-time quizzes with other users, with results displayed immediately.
\end{itemize}


\subsection{Identity, Nature and Characteristics of External Entities and Services and Data Used}

\subsection*{External Entities}
\begin{itemize}
    \item \textbf{Users (Students and Teachers):} Interact with EduPal to upload, manage, and access course content or take quizzes, utilise the features such as Pomodoro timer, quizzes, etc.
    \item \textbf{Object Storage System:} The external data storage system for EduPal. Stores all conspectus files.
\end{itemize}

\subsection*{Users}
\begin{itemize}
    \item Require a reliable and user-friendly interface.
    \item Access the system from various devices and locations.
    \item Depend on system availability and performance for effective learning.
\end{itemize}

\subsection*{Object Storage System}
\begin{itemize}
    \item High availability and reliability.
    \item Capable of handling large amounts of data.
\end{itemize}

\subsection*{Data Used}
\begin{itemize}
    \item \textbf{User Data:} Includes quiz related data, such as results.
    \item \textbf{Quiz Data:} Includes questions, answers, and results of the real-time quizzes.
\end{itemize}


\subsection{Identity and Responsibilities, and Nature and Characteristics of External Interfaces}

\subsection*{External Interfaces}
\begin{itemize}
    \item \textbf{User Interface:}
          \begin{itemize}
              \item \textbf{Identity:} The web-based interface accessed by users (students and teachers) via browsers on various devices.
              \item \textbf{Responsibilities:}
                    Users do not have any responsibilities.
          \end{itemize}
    \item \textbf{Object Storage System Interface:}
          \begin{itemize}
              \item \textbf{Identity:} The interface through which the backend communicates with the object storage system.
              \item \textbf{Responsibilities:}
                    \begin{itemize}
                        \item Store and retrieve conspectus files.
                        \item Ensure high availability, reliability, and data integrity.
                        \item Scale to accommodate growing data volumes.
                    \end{itemize}
          \end{itemize}
\end{itemize}

\subsection*{User Interface}
\begin{itemize}
    \item \textbf{Expected Volumes:}
          \begin{itemize}
              \item High volume of requests during peak hours such as during exam periods or assignment deadlines.
              \item Varying data sizes, from small interactions (e.g., note-taking) to larger uploads (e.g., course material).
              \item Growth expected as user base increases.
          \end{itemize}
    \item \textbf{Interaction Types:}
          \begin{itemize}
              \item Mostly ad hoc interactions initiated by users.
              \item Automatic topic archival is scheduled.
          \end{itemize}
    \item \textbf{Interaction Automation Level:}
          \begin{itemize}
              \item Almost all interactions are fully manual, except some that are in between manual and automated, such as the Pomodoro timer.
          \end{itemize}
    \item \textbf{Transactional Nature:}
          \begin{itemize}
              \item None of the interactions are fully transactional, as all of them can be cancelled.
          \end{itemize}
    \item \textbf{Criticality and Timeliness:}
          \begin{itemize}
              \item Timely responses critical for a smooth user experience.
          \end{itemize}
    \item \textbf{Interaction Mode:}
          \begin{itemize}
              \item Primarily message-based, with occasional batch operations (e.g., bulk uploads).
          \end{itemize}
    \item \textbf{Security Requirements:}
          \begin{itemize}
              \item High level of security required, including user authentication and data confidentiality.
          \end{itemize}
    \item \textbf{Service Levels:}
          \begin{itemize}
              \item High availability and low latency expected to ensure user satisfaction.
          \end{itemize}
    \item \textbf{Technical Nature:}
          \begin{itemize}
              \item Uses web technologies and open standards (e.g., HTTPS, REST API).
          \end{itemize}
    \item \textbf{Data and File Formats:}
          \begin{itemize}
              \item Supports various file formats for conspectus uploads (PDF) and quiz question images (PNG, SVG, JPEG, WEBP, GIF).
          \end{itemize}
\end{itemize}

\subsection*{Object Storage System Interface}
\begin{itemize}
    \item \textbf{Expected Volumes:}
          \begin{itemize}
              \item Average data volume of data transactions dependent on frequent user interactions with topic conspectuses.
              \item Expected growth as more data is accumulated.
          \end{itemize}
    \item \textbf{Interaction Types:}
          \begin{itemize}
              \item Automated interactions from backend services.
              \item A few scheduled interactions.
          \end{itemize}
    \item \textbf{Interaction Automation Level:}
          \begin{itemize}
              \item All operations are automated.
          \end{itemize}
    \item \textbf{Transactional Nature:}
          \begin{itemize}
              \item Interactions are transactional to ensure data integrity.
          \end{itemize}
    \item \textbf{Criticality and Timeliness:}
          \begin{itemize}
              \item High criticality for data retrieval and storage operations.
              \item Timeliness is only somewhat important as users can endure some loading times.
          \end{itemize}
    \item \textbf{Interaction Mode:}
          \begin{itemize}
              \item Entirely message-based.
          \end{itemize}
    \item \textbf{Security Requirements:}
          \begin{itemize}
              \item High security required, including data encryption.
          \end{itemize}
    \item \textbf{Service Levels:}
          \begin{itemize}
              \item High availability, scalability, and performance required.
          \end{itemize}
    \item \textbf{Technical Nature:}
          \begin{itemize}
              \item Standard Object Storage System protocols.
          \end{itemize}
    \item \textbf{Data and File Formats:}
          \begin{itemize}
              \item Data is stored in large binary objects.
          \end{itemize}
\end{itemize}


\subsection{Other External Interdependencies}

No other external interdependencies are identified.


\subsection{Impact of the System on Its Environment}

\begin{itemize}
    \item Improved learning management for students and teachers.
    \item Potential reduction in the usage of other learning systems if EduPal is more efficient.
    \item Students may migrate their conspectuses and other content to EduPal.
\end{itemize}


\subsection{Overall Completeness, Consistency, and Coherence}

\begin{itemize}
    \item Ensure all functionalities are well-integrated and user experiences are seamless.
    \item Maintain data consistency across all modules and interfaces.
    \item Plan for scalability and future integrations.
\end{itemize}


\subsection{Perspectives}


\subsection*{The Security perspective}


\subsection*{The Performance and Scalability perspective}


\subsection*{The Availability and Resilience perspective}


\subsection*{The Evolution perspective}


\subsection*{The Location perspective}


\subsection*{The Regulation perspective}


\subsection*{The Usability perspective}

\section{Functional viewpoint}

\subsection{Components}

\begin{figure}[H]
  \includedrawio{viewpoints/functional/components.drawio}
  \label{fig:viewpoint-functional-components}
  \caption{Component diagram}
\end{figure}

Every interface without explicitly specified protocol is done as a code interface with instance classes knowing about each other via dependency injection.

\begin{description}
  \item[User Web Browser:] Component responsible for communication with front-facing EduPal interfaces as well as displaying EduPal web pages. Web browser, being external and user controlled entity, is not trusted and its requests are verified by front-facing interfaces.
  \item[React Web Page Server:] Component responsible for serving EduPal React webpages to the web browser. This component does not have access to other parts of the EduPal system and is only responsible for delivery of hypertext, media and code assets that are executed and rendered on the web browser.
  \item[API Service:] Component responsible for handling user requests, validating them and communicating with functionality related services (Quiz management, Topic management) to perform the operation and prepare the response.
  \item[Auhtorization Service:] Component responsible for determining whether certain type of request can be processed (checks whether user has the privilege) and whether user has the right to perform certain operations (such as quiz creation).
  \item[Topic and Quiz Management Services:] Components that API Service is communicating with. These components offer creation, update, delete and other data management commands and are responsible for checking if operation can be performed based on business rules.
  \item[Data Storage components:] Components responsible for abstracting data access and update logic. Quiz Game Session Data Storage is extracted as a separate component as a separation of concerns because quiz game sessions are short, require frequent updates and are unrelated to the rest of EduPal system.
  \item[Quiz Game Session Hub:] Component responsible for maintaining SignalR communication with the web browser, performing identity check via authorization service and passing requests to Quiz Game Session Handler as well as reacting to Quiz Game Session Handler events.
  \item[Quiz Game Session Handler:] Component responsible for handling the quiz game business logic, grading answers and communicating with the data storage.

\end{description}

\section{Information viewpoint}

\subsection{Information Structure and Content}

\begin{figure}[H]
  \includedrawio[width=\textwidth]{viewpoints/information/entities.drawio}
  \caption{Persistent entity diagram}
  \label{fig:viewpoint-information-entity}
\end{figure}

\subsection{Information Purpose and Usage}

All presented entities are required for EduPal operation with quiz functionality. User, Topic, Quiz, Question and Option entities are responsible for static data storage, while Quiz Game Session, Quiz Game Player and Quiz Game Player Answer are used for multiplayer quizzes. Entities are used only for OLTP processing as no data analysis requirements exist. Purposes and caveats of certain entities and relations:

\begin{enumerate}
  \item Topic may have the owner and quiz has the topic - required to trace who is the owner of a particular quiz.
  \item Quiz Game Session stores pointers to shuffled questions (it is expected that implementation will be using id's of questions) to ensure that order of questions in every game is different.
  \item Quiz Game Player is a separate entity of user to decrease coupling between entities and ensure that possible future requirement of anonymous players will be possible to implement without complex changes.
  \item Question may point to an image URI stored in external object storage system.
  \item Quiz Game session is only stored while session is active.
  \item Last access field is need for archivation purposes in the future - entities that have not been used for a long time will move to future archive database.
\end{enumerate}

\subsection{Information Ownership}

EduPal uses a single centric database to simplify operations. This ensures that we do not have to deal with information ownership issues.

\subsection{Identifiers}

Every entity must have an explicit Id attribute with unique (for the entity group) UUID value. Id attribute acts as a primary key and cannot change throughout the operation. The decision to have explicit primary key attribute and not to use some other data (such as username) as a primary key ensures that every entity has a similar structure and constraints of other attributes can be changed (username may or may not be unique). Every image has a unique URI and it is the responsibility of external object storage system to ensure this.

\subsection{Volatility of Information Semantics}

Database schema and its migrations are tracked by Entity Framework Core ORM. Since the database is only accessed by the monolithic EduPal application and none of the other systems, this management option is enough and allows to save developer time and cut down unnecessary costs.

\subsection{Information Storage Models}

Static data will be stored in a relational database - this ensures data consistency and allows to specify proper constraints for the data. Images will be stored in external object storage system (that is optimized for quick access and upload of 1-100Mb files).

Quiz game sessions require fast data reads and writes and are deleted upon completion. To achieve high performance NoSQL document or key-value database with efficient caching capabilities should be used. A support for in-memory operation would also be beneficial.

\subsection{Information Flow}

Information is created, accessed, destroyed and modified by the business layer of the EduPal.

\subsection{Information consistency}

In order to increase performance during request processing EduPal chooses to use eventual consistency model - data access layer may cache certain data changes and not update the database in case the entity is commonly queried or updated.

\subsection{Archiving and Information Retention}

EduPal is still a young and developing system without a huge amount of data. Because of that and not a huge amount of information stored (Every question is typically one sentence, option is a few words), we currently do not need any automatic or even manual archivation processes. In the future, the data access layer will need to be extended - we will add archive database that old entities will go to. Data access layer and business layer will need to be updated to move information between databases.

\subsection{Security perspective}

Every static information is publicly available to the end users, therefore the only resources we need to securely store are user credentials. Because of low amount of entities with high sensitivity, we do not think it is necessary to separate data into a separate database. Data integrity is ensured by the fact that database is only accessed and updated by the EduPal system. Since every entity has a single owner, every data state change can be traced to the user.

\subsection{Performance and Scalability}

In order to increase performance, quiz sessions are stored in a NoSQL database. Scalability is primarily limited by a monolithic system architecture and for such architecture single database is enough.

\subsection{Availability and recovery}

In case of a controllable system crash (such as unhandled exception or similar), before the shutdown the system data access layer will perform the synchronization with the database. In case of a serious and uncontrollable crash (hardware failure or recovery crash and similar) temporary data will be lost. While this is a critical issue, considering the scope of EduPal the scenario is acceptable. Additionally, to minimize the losses, data access layer must synchronize the changes that are at least one hour old.

\subsection{Evolution perspective}

Information model is flexible as entities are related only when necessary and only required attributes are stored ensuring that the information model may evolve as need. Additionally, Some entities (such as Quiz Game Player and User) are decoupled to minimize future changes in case of new requirements.

\subsection{Location perspective}

Unrelated, since data is not distributed.

\subsection{Regulation perspective}

Only necessary data that is identified by terms of service is collected. EduPal does not archive data of deleted users and instead completely deletes the data entries (in case of user deletion it means that their topics and quizzes will be deleted). Data that is necessary only during quiz session is deleted by the end of it.

\subsection{Usability perspective}

High information quality and properly defined relations make the information highly usable and convenient to operate in business layers.

\documentclass{article}
\usepackage{graphicx}
\usepackage{enumitem}

\begin{document}

\section{Development View}

\subsection{Module Organization}

The EduPal quiz integration feature is organized into several interconnected modules, each responsible for specific aspects of the system:

\subsubsection{User Interface Module}

\begin{itemize}
  \item \textbf{QuizPage}: The main page where users can start the quiz.
  \item \textbf{QuestionPage}: The page that displays quiz questions to the users.
  \item \textbf{ResultPage}: The page that shows the results after the quiz is completed.
\end{itemize}

\subsubsection{Quiz Session Module}

\begin{itemize}
  \item \textbf{QuizSessionController}: Manages user interactions during a quiz session.
  \item \textbf{QuizSessionService}: Handles the business logic associated with a quiz session.
  \item \textbf{QuizSessionRepository}: Manages the data storage and retrieval for quiz sessions.
\end{itemize}

\subsubsection{Quiz Management Module}

\begin{itemize}
  \item \textbf{QuizController}: Manages the interactions related to quiz management.
  \item \textbf{QuizService}: Contains the business logic for quiz management.
  \item \textbf{QuizRepository}: Responsible for storing and retrieving quiz-related data.
\end{itemize}

\subsubsection{Question Bank Module}

\begin{itemize}
  \item \textbf{QuestionController}: Manages interactions related to question management.
  \item \textbf{QuestionService}: Contains the business logic for question management.
  \item \textbf{QuestionRepository}: Responsible for storing and retrieving question data.
\end{itemize}

\begin{figure}[htbp]
    \centering
    \includedrawio[width=\textwidth]{viewpoints/development/development.drawio}
    \caption{Module Organization Diagram}
    \label{fig:development}
\end{figure}


\subsubsection{Common Processing}

\item \textbf{Data Validation:} Uses React with Formik and Yup for form handling and schema-based validation.

\item \textbf{Error Handling:} Uses ASP.NET Core middleware for global exception handling, logging errors to a centralized logging service (like Serilog) and returning standardized error responses.

\item \textbf{User Authentication:} Manages user authentication and authorization using JWT tokens for securing API endpoints. Middleware checks token validity and user roles before processing quiz-related requests.

\subsection{Standardization of Design}

\subsubsection{Design Patterns}

\begin{enumerate}[label= \arabic*.] \item \textbf{Factory Pattern:} \begin{itemize}[label=$\bullet$] \item \textbf{Purpose:} Used for creating instances of different quiz types. \item \textbf{Implementation:} Utilize a \texttt{QuizFactory} class responsible for encapsulating the instantiation logic based on the provided quiz type. \end{itemize}

\item \textbf{Repository Pattern:} \begin{itemize}[label=$\bullet$] \item \textbf{Purpose:} Abstracts data access and creates a clean separation between the business logic and data access layers. \item \textbf{Implementation:} All repository classes must adhere to a common interface, named \texttt{IBaseRepository}, which defines methods for CRUD operations. \item \textbf{Interface Requirement:} Any new repository must implement the \texttt{IBaseRepository} interface. \end{itemize}

\item \textbf{Singleton Pattern:} \begin{itemize}[label=$\bullet$] \item \textbf{Purpose:} Ensures a single instance of certain services, like \texttt{QuizSessionService}, for managing application-wide state. \item \textbf{Implementation:} Implement the Singleton pattern to guarantee that only one instance of the service is created and accessed throughout the application's lifecycle. \end{itemize}

\end{enumerate}

\subsubsection{Logging Package and Usage}

\begin{itemize}[label=$\bullet$] \item \textbf{Logging Package:} Consistently use a logging framework like Serilog. \item \textbf{Configuration:} Ensure the presence of a logging configuration file (e.g., \texttt{serilog.json}) in the application's directory. This file describes logging behavior, such as log levels and output destinations. \item \textbf{Usage in Code:} Log messages using the logger obtained from the logging framework's logging API. This logger should be declared as a private static readonly field within the class. \end{itemize}

\subsubsection{Repository Implementation}

\begin{itemize}[label=$\bullet$] \item \textbf{Interface Requirement:} All repository implementations must adhere to the common interface. \item \textbf{Consistency:} Maintain consistency in method signatures and behavior defined by the \texttt{IBaseRepository} interface. \end{itemize}

\subsubsection{Dependency Injection}

Dependency Injection (DI) is heavily employed throughout our system. By facilitating loose coupling between components. 

\subsubsection{ Layered Structure and Module Organization}

To enhance organization and clarity, we advocate for a structured approach to module organization:

\begin{itemize}[label=$\bullet$] \item \textbf{Interface Module:} This module serves external interactions and communication, providing a clear interface for external systems or users to interact with our application.

  \item \textbf{Core Module (Business Logic):} At the heart of our system lies the core module, responsible for orchestrating operations and enforcing business rules. This module encapsulates the essential logic driving our application's functionality.

  \item \textbf{Data Access Layer:} The data access layer provides an interface for interacting with data storage, ensuring separation of concerns and facilitating efficient data management. \end{itemize}

\subsubsection{Consistent User Flows}

  \item \textbf{Quiz Taking Flow:} Users navigate from the quiz list to the quiz page, answer questions sequentially, and then view results. Each step is guided by consistent visual cues and navigation buttons.

  \item \textbf{Quiz Management Flow:} Admins follow a flow from creating a new quiz, adding questions, setting correct answers, and publishing the quiz.

\subsection{Standardization of Testing}

\subsubsection{Unit Testing}
For our backend unit testing, we use NUnit and Moq for testing ASP.NET Core services and repositories. These tests cover validation logic, business rules, and data access methods, focusing on service methods and repository functions. They follow the Arrange-Act-Assert pattern and are organized in a dedicated test project that mirrors the main codebase structure. Mock objects are used to simulate dependencies, and cleanup functions reset the state between tests to ensure isolation and reliability of test results.

\subsubsection{Integration Testing}
For API endpoints, we use Postman to automate comprehensive integration tests. Postman scripts verify responses to various HTTP requests (GET, POST, PUT, DELETE). These tests are organized in collections and executed across multiple environments.

We use Cypress for detailed end-to-end tests. For example, in our upcoming quiz editor implementation, the following test scenarios are covered:

\begin{itemize}
    \item \textbf{Quiz Editor Access:} The test runner opens the quiz management page and verifies that the quiz edit button is visible for the topic owner and not visible for non-owners.
    \item \textbf{Quiz Creation and Editing:} The test runner navigates to the quiz creation page, enters quiz details (e.g., title, questions, answers), and submits the form. It then checks the backend response to confirm the quiz was successfully created. The runner subsequently opens the quiz edit page, modifies quiz details, submits changes, and confirms updates were successful.
    \item \textbf{Quiz Deletion:} The test runner clicks the delete button for a quiz, confirms the action, and checks the backend response.
\end{itemize}

These tests are stored in a cypress directory and focus on critical workflows, such as user authentication, role-based access, form submissions, data validation, and ensuring data passes correctly between the frontend and backend.

\subsection{Instrumentation}

\item \textbf{Monitoring Performance and Usage:}

\item \textbf{Logging:} Incorporates Serilog for structured logging. Logs include details on user interactions with quizzes, errors, and performance metrics.

\item \textbf{Analytics:} Uses Google Analytics to track user behavior, quiz engagement rates, and completion statistics. Custom events track specific actions like quiz start, question answer, and quiz completion.

\item \textbf{Usage Metrics:} Stores data on question difficulty, user scores, and completion times in a PostgreSQL database. Regular reports are generated to identify trends and areas for improvement.

\subsection{Codeline Organization}

\textbf{Modular Structure:}

\item \textbf{Frontend:} Organized into directories for components (/components), pages (/pages), services (/services), and utilities (/utils). Each component and page related to the quiz feature has its own directory.

\item \textbf{Backend:} Follows a similar structure with directories for controllers (/Controllers), services (/Services), repositories (/Repositories), and models (/Models). The quiz feature-specific code is encapsulated within its own subdirectories.

\item \textbf{Branching Strategy:} Uses Git with a feature branch workflow.

\subsection{Stakeholder Concerns}

\subsubsection{Developers}

\begin{itemize}
  \item Need clear module boundaries and documentation to facilitate development and maintenance.
  \item Require standardized testing and instrumentation practices to ensure code quality and performance.
\end{itemize}

\subsubsection{Production Engineers}

\begin{itemize}
  \item Concerned with the reliability and scalability of the quiz feature.
  \item Need comprehensive logging and monitoring to quickly identify and resolve issues.
\end{itemize}

\subsubsection{Testers}

\begin{itemize}
  \item Require detailed test plans and automated testing scripts to efficiently validate the system.
  \item Benefit from standardized testing approaches to ensure consistent test coverage and results.
\end{itemize}

\end{document}

\listoffigures

\end{document}
