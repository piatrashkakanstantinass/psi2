\documentclass[
    english, % Klasei padavus parametrą 'english', darbas bus anglų kalba.
    % signatureplaces % prideda parašų vietas tituliniame puslapyje
]{VUMIFPSkursinis}
\usepackage{float}
\usepackage{wrapfig2}
\usepackage{hyperref}
\usepackage{algorithmicx}
\usepackage{algorithm}
\usepackage{algpseudocode}
\usepackage{amsfonts}
\usepackage{amsmath}
\usepackage{bm}
\usepackage{caption}
\usepackage{color}
\usepackage{graphicx}
\usepackage{listings}
\usepackage{subcaption}
\usepackage{biblatex}
\usepackage{geometry}
\usepackage{booktabs}
\usepackage{multirow}
\usepackage{diagbox}
\usepackage{afterpage}
\usepackage{makecell}
\usepackage[inkscapelatex=false]{svg}
\renewcommand{\cftdotsep}{1} 


\university{Vilnius university}
\faculty{Faculty of mathematics and informatics}
\department{Software engineering study program}
\papertype{Software Engineering II laboratory work 2}
\title{EduPal change request analysis, requirements and project plan}
\status{2 course 5 group students}
\author{Motiejus Šveikauskas}
\secondauthor{Kanstantinas Piatrashka}
\thirdauthor{Aldas Vertelis}
\fourthauthor{Danielius Podbielski}
\reviewer{doc. dr. Vardauskas Pavardauskas}
\date{Vilnius – \the\year}

\bibliography{bibliografija}

\begin{document}
\maketitle

\tableofcontents

\section{Context}

EduPal is a learning aid business which helps both students and teachers by allowing educators to conveniently upload course information for students to learn. EduPal offers unique features such as pomodoro timer, note creation and export in pdf, setting goals, chatGPT integration. EduPal lets users have subjects, create topics inside subjects and upload conspectuses.

\subsection{Change request}

\textbf{User need:} Users want to revise learned material in interactive and efficient way. We propose an idea to integrate quizzes into the system. Every topic should have a quiz section where topic owner  should be able to create, update and delete quizzes. Quiz consists of randomly shuffled questions, question answer interface should be simple and easy to use with mouse or touch device. Upon answering a question, user gets an instant feedback.

\vspace{\baselineskip}

Currently EduPal system offers no way to revise or test knowledge of a particular topic. It is responsibility of users to check their own understanding of material by using their own approach which takes more time than they would like and adds a complexity of choosing a third party service or methodology. Integrating quizzes into a system would boost engagement of users and simplify their studying process.

\subsection{Change priority}

Majority of users need to use EduPal with third party active learning tool that provides feedback (typically flashcard or quiz platform). Many major players (such as Quizlet and Kahoot) do not yet have proper conspectus storage system such as EduPal, meaning that users of those platforms tend to rely on solutions such as EduPal. Implementing quizzes can motivate active learning tool users to try EduPal while some may decide that only EduPal is enough for their needs. Additionally, active learning tool platforms may implement data storage and sharing system similar to EduPal that would result in users migrating away from EduPal. Benefits of built in quiz capability include:

\begin{enumerate}
    \item EduPal can be treated as all-in-one system
    \item Users will not need to search for and configure third party systems
    \item EduPal can be advertised to a broader user base.
    \item Quizzes would also provide valuable data insights into user engagement and performance, which could be used for strategic decisions by managers.
\end{enumerate}

And drawbacks are:

\begin{enumerate}
    \item Being all-in-one system, EduPal will have to compete with both conspectus storage systems and active learning tool systems which requires more financial investments
    \item Some users will not want to use EduPal solution as they are accustomed to their setups
\end{enumerate}

We consider this change priority to be \textbf{medium to high}.

\subsection{Affected stakeholders}

The feature would affect these stakeholders:

\begin{enumerate}
    \item \textbf{Teachers} would get the opportunity to emphasize more important questions by creating quizzes and would have the chance to provide more helpful material in a different format.
    \item \textbf{Students} would have the opportunity to learn in an easier and more useful way. The opportunity to test their own knowledge would allow students to realize which topics they know well and on which topics they should revise.
    \item \textbf{Developers and Maintainers} would have to create, test and maintain the feature.
    \item \textbf{Marketers and Support} would have to advertise the feature and help users as well as collect bug reports and future functionality requests for developers and business analysts.
    \item \textbf{Business owners and Investors} will likely see increase in user base and profits.
\end{enumerate}

\subsection{Implementation alternatives}

\subsubsection{First option: use notes to provide links to third party quiz service}

This simple implementation alternative requires no implementation whatsoever.  We would suggest teachers to create a note with a link to quiz on another website, where a new user account would have to be created if the user is using the app for the first time, otherwise logging in is sufficient. Students would be able to list the notes and pick a quiz they like.

\textbf{Correctness:} \textbf{Low}. We force users to do work that could be automated if we picked a more costly solution. Low user satisfaction is expected if we pick this alternative. The disadvantage of this approach is that both the teachers and the students may be forced to create a new account in a separate application. Additionally, different users may prefer different quiz applications resulting in confusion.

\textbf{Duration:} \textbf{~1 day}. EduPal will only need to suggest using platform in this way in social media. It is enough to make few posts and potential users will know about the possibility.

\textbf{Pricing:} \textbf{None}. This option does not involve any cost from EduPal side, however, users who the quizzes may need to pay for premium account feature on certain quiz applications. Our research suggests that majority of popular quiz websites provide extensive set of features for free.

\textbf{Extendability:} \textbf{Not extendable}. Quiz functionality is not controlled by EduPal.

\textbf{Maintenance:} \textbf{None} We do not have to maintain it as it falls upon the owners of the third-party quiz app, saving us money and time as a result.

\subsubsection{Second option: integrate third-party API}

Our team proposes integrating a third-party quizzing app into our system through an API or by embedding it, perhaps even establishing a potential partnership with a suitable quizzing app. Users would be able to create and take quizzes without leaving EduPal website, while quiz data will be stored by third-party.

\textbf{Correctness:} \textbf{Medium-High}. Using an external API is considered to be a valid and acceptable choice in the industry. Using third party embedded frames would be convenient to use, but may not be uniform with EduPal website style. Using the API, and implementing UI ourselves will result in similar user experience as if the feature was completely implemented by EduPal.

\textbf{Duration:} \textbf{1-3 months}, including time for developing quiz buttons, understanding the API, and establishing potential partnerships.

\textbf{Pricing:} \textbf{5-10k euros}. This option requires finding 

\textbf{Extendability:} \textbf{Low}, as we lack control over the third-party API, restricting our ability to make extensive changes.

\textbf{Maintenance:} \textbf{Low-Medium}. Increased maintenance demands due to shared functionality ownership and the need to keep up with API changes.

\subsubsection{Third option: add quiz subsystem to EduPal}

Third option involves our team implementing a fully fledged quizzing feature in the EduPal system. In this option our team would develop the quiz app embedded in EduPal with its own API and frontend choices.

\textbf{Correctness:} \textbf{High}. Implementing quiz subsystem results in a fluid system that has room for new features and integration. Users will experience convenient UI and will not have to navigate to external pages.

\textbf{Duration:} \textbf{1-5 months} for full implementation, including development and testing.

\textbf{Pricing:} \textbf{>10k euros}. This option is the most costly due to the need for a dedicated team of developers, designers and testers.

\textbf{Extendability:} \textbf{High}. Since EduPal team will maintain the quiz subsystem, it will be possible to implement additional features and enable deeper integration with other parts of EduPal.

\textbf{Maintenance:} \textbf{High}. The addition of a new service increases maintenance challenges as we will have to handle discovering and fixing bugs, maintain a dedicated infrastructure, ensure that implementation does not impose a technical debt on other parts of EduPal system.

\subsection{Implementations comparison}

\begin{figure}[ht]
    \centering
    \begin{tabular}{|c|c|c|c|}
        \hline
                                & First option & Second option & Third option \\
        \hline
        Correctness             & Low         &Medium-High         &  High            \\
        \hline
        Duration                & 1 day        & 1-3 months     & 1-5 months   \\
        \hline
        Cost                    & 0            & 5-10k euros   & >10k euros   \\
        \hline
        Extendability           & None      & Low    & High    \\
        \hline
        Maintenance             & None          & Low-Medium           & High          \\
        \hline
    \end{tabular}
    \caption{Implementations comparison}
    \label{tab:implementations-comparison}
\end{figure}

\subsection{Picked alternative}

EduPal decided to pick third option and implement the quiz functionality ourselves. Quiz functionality is important to users and they wont be satisfied with the workaround like first option, leaving us with two similar considerations: integrate third-party API or write our own subsystem. Both options are costly and take a similar period of time to complete, however implementing quizzes ourselves allows us to:

\begin{itemize}
    \item Integrate quizzes deeper with other EduPal functionality.
    \item Extend quiz functionality later.
    \item Be independent from third party that can shutdown or impose undesirable conditions.
\end{itemize}

\subsection{Outsourcing capability}

Since quiz subsystem will need to integrate with existing EduPal system, architecture and development are not cost efficient to outsource. However, we may need designers and testers that could be outsourced.

\subsection{Project management}

We put our business team as core management of the project. They must ensure that the product acknowledges end user needs, control the budget and execution time, handle advertisement of the feature. Development team must follow business team decision, however developers need to ensure that quiz integration will not have performance or maintenance impact on other parts of the system and will not result in technical debt.

\section{Domain model}

\begin{figure}[ht]
    \centering
    \includegraphics[width=\textwidth]{../lab3diags/domain.drawio.png}
    \caption{Domain model diagram}
    \label{domain-model}
\end{figure}

\begin{description}
    \item[User] Every EduPal customer is user - they are able to create subjects, topics, upload conspectuses, rate them and so on.
    \item[Topic owner] User who owns a particular topic - only they are able to edit its contents, including quizzes.
    \item[Topic] A part of subject (not related to change request). Topic consists of conspectuses, comments and quizzes.
    \item[Quiz] A named collection of questions
    \item[Question] Question has a name and a set of incorrect options as well as correct one.
    \item[Option] Question option is a text answer.
    \item[Quiz session] Users start quiz session when they take the quiz.
    \item[User answer] User answer is an option that was picked by user.
\end{description}

\section{Use cases}

\subsection{List quizzes}

\begin{figure}[ht]
    \centering
    \includegraphics[width=\textwidth]{../lab3diags/list.drawio Large.jpeg}
    \caption{List quizzes use case}
    \label{list-quizzes}
\end{figure}

\begin{figure}[ht]
    \centering
    \includegraphics[width=\textwidth]{../lab3diags/Quiz list (User).png}
    \caption{User list of quizzes wireframe}
    \label{wireframe-quiz-list-user}
\end{figure}

\begin{figure}[ht]
    \centering
    \includegraphics[width=\textwidth]{../lab3diags/Quiz list.png}
    \caption{Topic owner list of quizzes wireframe}
    \label{wireframe-quiz-list-topic-owner}
\end{figure}

\noindent\textbf{\fontsize{13}{15}\selectfont Main scenario:}

\textit{The user} opens the EduPal website; the system displays the Login Page. \textit{The user} enters login information and clicks the Login Button; the system verifies identity of \textit{the user}, initiates the user session, and shows the Subject List Page. \textit{The user} clicks on a specific subject; the system loads its \textit{topics} and shows the Topic List Page for that subject. \textit{The user} clicks on a specific topic; the system loads \textit{topic} data and shows the Topic Page. \textit{The user} clicks the Quizzes Button; the system fetches a list of \textit{quizzes} associated with \textit{the topic} and shows a Quiz List page. \textit{The user} clicks on the particular \textit{quiz}; the system displays the Quiz Page.

\noindent\textbf{\fontsize{13}{15}\selectfont Alternatives:}

\textbf{\textit{The user} enters incorrect login information:} The system displays the Login Failed Message.

\textbf{\textit{User} session expired:} The system navigates \textit{the user} to the Login Page.

\textbf{\textit{The user} is \textit{the topic owner}} The system shows the Create Quiz button on the Quiz List Page. The system shows Edit and Delete Buttons on the Quiz Page.

\pagebreak

\subsection{Take the quiz}

\begin{figure}[ht]
    \centering
    \includegraphics[width=0.5\textwidth]{../lab3diags/take-quiz.drawio.png}
    \caption{Take the quiz use case}
    \label{take-quiz}
\end{figure}

\begin{figure}[ht]
    \centering
    \includegraphics[width=\textwidth]{../lab3diags/Quiz play (User).png}
    \caption{Taking the quiz}
    \label{wireframe-quiz-play}
\end{figure}

\begin{figure}[ht]
    \centering
    \includegraphics[width=\textwidth]{../lab3diags/Quiz completed (User).png}
    \caption{Feedback after taking the quiz}
    \label{wireframe-quiz-completed}
\end{figure}

\noindent\textbf{\fontsize{13}{15}\selectfont Main scenario:}

\textit{The user} is on the particular Quiz Page. \textit{The user} clicks the p
Play Button; the system creates a \textit{quiz session} with \textit{the user}, shuffles \textit{the questions}, and displays the Question Page. \textit{The user} clicks on \textit{an option}; the system records \textit{an answer} and shows the next \textit{question}. Once \textit{the user} answers all the \textit{questions}, the system calculates the percentage of correctly answered \textit{questions} and shows the Quiz Completed Page. \textit{The user} clicks the Close Button; the system shows the Quiz Page.

\noindent\textbf{\fontsize{13}{15}\selectfont Alternatives:}

\textbf{\textit{User answer} was correct:} System displays Correct Answer Notification for 5 seconds.

\textbf{\textit{User answer} was wrong:} The system displays the Wrong Answer Page. \textit{The user} clicks on the Continue Button; and the system proceeds with the next \textit{question}.

\textbf{\textit{The user} clicks on the Restart Button on the Quiz Completed Page:} The system reshuffles \textit{the questions}, starts a new \textit{quiz session}, and shows the Question Page.

\textbf{\textit{The user} tries to leave the Question Page:} The system displays the Confirmation Popup; if \textit{the user} confirms, the system closes the page and deletes \textit{the quiz session}.

\textbf{\textit{The quiz} was edited or deleted during \textit{the session}:} The system does not show a Restart Button on the Quiz Completed Page.

\begin{figure}[ht]
    \centering
    \includegraphics[width=0.7\textwidth]{../lab3diags/topic-owner-actions.drawio.png}
    \caption{Topic owner use cases}
    \label{topic-owner}
\end{figure}

\subsection{Delete the quiz}

\begin{figure}[ht]
    \centering
    \includegraphics[width=\textwidth]{../lab3diags/Quiz delete popup.png}
    \caption{Confirmation popup for deleting the quiz}
    \label{wireframe-quiz-delete-popup}
\end{figure}

\noindent\textbf{\fontsize{13}{15}\selectfont Main scenario:}

\textit{The topic owner} is on the particular Quiz Page. \textit{The topic owner} clicks on the Delete Quiz Button; the system shows a Deletion Confirmation Popup. \textit{The topic owner} clicks on the Confirm Deletion Button; the system deletes \textit{the quiz}, shows the Quiz List Page for \textit{the topic} that the deleted \textit{quiz} belonged to, and displays a Quiz Deleted Notification.

\noindent\textbf{\fontsize{13}{15}\selectfont Alternatives:}

\textbf{\textit{Topic owner} closes Confirmation Popup:} System aborts deletion.

\subsection{Create the quiz}

\begin{figure}[ht]
    \centering
    \includegraphics[width=\textwidth]{../lab3diags/Quiz creation.png}
    \caption{Creating the quiz}
    \label{wireframe-quiz-creation}
\end{figure}

\begin{figure}[ht]
    \centering
    \includegraphics[width=\textwidth]{../lab3diags/Quiz creation error.png}
    \caption{Error popup when creating the quiz}
    \label{wireframe-quiz-creation-error}
\end{figure}

\noindent\textbf{\fontsize{13}{15}\selectfont Main scenario:}

\textit{The topic owner} is on the particular Quiz Page. \textit{The topic owner} clicks the Create Quiz Button; the system creates \textit{the empty quiz} in memory and shows the Quiz Editor Page associated with \textit{the quiz}. \textit{The topic owner} fills in \textit{the quiz} information and clicks the Save Quiz Button; the system saves \textit{the quiz} and shows the Quiz Page for that \textit{quiz}.

\noindent\textbf{\fontsize{13}{15}\selectfont Alternatives:}

\textbf{\textit{The topic owner} tries to leave the Quiz Editor Page:} The system displays the Confirmation Popup; if \textit{the topic owner} confirms, the system shows the Quiz List Page and removes \textit{the quiz} from memory.

\subsection{Edit the quiz}

\noindent\textbf{\fontsize{13}{15}\selectfont Main scenario:}

\textit{The topic owner} is on the particular Quiz Page. \textit{The topic owner} clicks on the Edit Button; the system shows the Quiz Editor Page associated with \textit{the quiz}. \textit{The topic owner} fills in the \textit{quiz} information and clicks the Save Quiz Button; the system saves \textit{the quiz} and shows the Quiz Page for that \textit{quiz}.

\noindent\textbf{\fontsize{13}{15}\selectfont Alternatives:}

\textbf{\textit{The topic owner} tries to leave the Quiz Editor Page:} The system displays the Confirmation Popup; if \textit{the topic owner} confirms, the system shows the Quiz List Page and aborts the changes.

\subsection{Quiz editor use cases}

Use cases here assume that \textit{the topic owner} is on the Quiz Editor Page associated with certain \textit{quiz}.

\subsubsection{Edit the quiz}

\noindent\textbf{\fontsize{13}{15}\selectfont Main scenario:}

The system shows the Quiz Editor Page. \textit{The topic owner} enters \textit{quiz} name in Quiz Name Field; system checks quiz name validity. \textit{The topic owner} clicks on the Add Question Button to add \textit{question}, the system adds a \textit{question} with no name, no image, and 2 empty \textit{options} to \textit{the quiz} and adds a Question Editor Component to the Question List on the Quiz Editor. \textit{The topic owner} clicks on the Remove Question Button to remove \textit{the question}, the system removes \textit{the question} and removes the Question Editor Component Associated with that \textit{question}. \textit{The topic owner} edits \textit{the questions} and clicks the Save Quiz Button; the system performs the action defined in the invoking use case.

\noindent\textbf{\fontsize{13}{15}\selectfont Alternatives:}

\textbf{Bad input: \textit{Quiz} has no \textit{questions}:} System shows No Questions Message near Add Question Button.

\textbf{Bad input: Text in text fields (both \textit{quiz} name and Question Components) is shorter than 5 characters or longer than 50:} The system shows a name error message near the affected text field.

\textbf{There are bad inputs:} The system disables the Save Quiz Button.

\subsubsection{Edit the question}

The correct \textit{option} is \textit{the option} on top of the Option List.

\noindent\textbf{\fontsize{13}{15}\selectfont Main scenario:}

\textit{The topic owner} enters the \textit{question} name in the Question Name Field and \textit{option} names in the Option Name Fields; the system checks the validity of the entered information. \textit{The topic owner} clicks on the Upload Image Button; the system opens the file picker. \textit{The topic owner} chooses an image file; the system downloads the image, saves it, displays the thumbnail and replaces the Add Image Button with Remove Image Button. \textit{The topic owner} edits the options; system acts as defined in \textit{Question editor: edit options (\ref{question-editor})}

\noindent\textbf{\fontsize{13}{15}\selectfont Alternatives:}

\textbf{\textit{The topic owner} wants to remove the image:} \textit{The topic owner} clicks on the Remove Image Button; the system deletes the image, removes it from the Question Component and replaces the Remove Image Button with the Add Image Button.

\subsubsubsection{Question editor: edit options} \label{question-editor}

\noindent\textbf{\fontsize{13}{15}\selectfont Main scenario:}

If \textit{the topic owner} wants to add an \textit{option}, they click on the Add Option Button; the system adds a new empty \textit{option} at the end of the Option List. If \textit{the topic owner} wants to remove an \textit{option}, they click on the Remove Option Button; the system removes \textit{the option}. If \textit{the option} was marked as the correct one (was on top of the list), system makes the second \textit{option} correct. If \textit{the topic owner} wants to change the correct \textit{option} to some other \textit{option}, they drag \textit{the option} to the top of the Option List; the system marks the dragged \textit{option} as the correct one.

\noindent\textbf{\fontsize{13}{15}\selectfont Alternatives:}

\textbf{Bad input: 2 or more options have same text:} The system shows the Duplicate Options Error Message in the Question Component.

\textbf{There are 4 \textit{options}:} The system does not show the Add Option Button.

\textbf{There are 2 \textit{options}:} The system does not show the Remove Option Button

\section{Requirements}

\subsection{Functional requirements}

\newlist{frlist}{enumerate}{1}
\setlist[frlist]{label=\textbf{FR\arabic*.}, align=left, leftmargin=*}

\begin{frlist}
    \item System shall allow users to list quizzes.
    \item System shall ensure that only logged in users can access quizzes.
    \item System shall only allow creation, editing and deletion of the topic quizzes to owners of that topic.
    \item System shall support users taking the quiz.
    \item System shall check whether user answer is correct and display appropriate feedback.
    \item System shall ensure that users cannot restart the quiz that was edited or deleted during the quiz session.
    \item System shall allow topic owner to delete any quiz that belongs to their topic.
    \item System shall allow topic owner to create the quiz in their topic.
    \item System shall allow topic owner to edit any quiz in their topic.
    \item System shall prompt for confirmation when deleting quizzes.
    \item System shall support uploading image for questions.
    \item System shall support setting quiz name.
    \item System shall support adding, editing and deleting quiz questions and options.
\end{frlist}

\subsection{Nonfunctional requirements}

\newlist{nfrlist}{enumerate}{1}
\setlist[nfrlist]{label=\textbf{NFR\arabic*.}, align=left, leftmargin=*}

\begin{nfrlist}
    \item System shall support PNG, SVG, JPEG, WEBP, GIF question image formats.
    \item System shall ensure that quiz functionality related pages are supported by common screenreaders (VoiceOver, Microsoft Narrator, Jaws).
    \item System shall ensure that quiz functionality related pages are displayed on desktop and mobile without horizontal scrollbars or loss of information.
    \item System shall ensure that image upload and download speed is at least 500Mbit/s in common locations (Vilnius, Berlin, New York, Paris)
\end{nfrlist}

\section{Project plan}

\subsection{Team}

Execution team consists of four full-stack developers: Kanstantinas, Motiejus, Aldas, and Danielius. All have a variety of technical skills in both back-end and front-end development. All of them have experience and the right skills for architectural system design, User Interface design, database management, system deployment, and monitoring.

\subsection{Tasks}

\begin{enumerate}[label=\arabic*.]
    \item System architecture: Process of determining architecture of the quiz subsystem, its integration with EduPal architecture, component design, exposed API and similar architectural tasks.
    \item Infrastructure plan: Creating deployment plan, resource management plan, deciding on host platform (cloud or physical), container orchestration plan.
    \item User interface plan: UI \& UX design, creation of user flow charts, optimizing interface for mobile, deciding on how voice assistant should behave.
    \item Documentation: Writing of manuals, guides, code documentation detailing the system's functionality, architecture, installation, configuration, and usage for developers, administrators, and end-users.
    \item Frontend programming: Implementation of User Interface Plan.
    \item Backend programming: Implementation of System architecture.
    \item Test writing: Creation of unit, integration and regression tests.
    \item Error analysis: Discovery of errors in logic with the help of automated tests and manual testing. Process of documenting the errors.
    \item Error fixing: The process of resolving identified errors.
    \item Infrastructure setup: Process of setting up deployment target platform, preparing system for containerization, working with Docker and Kubernetes.
    \item Deployment implementation: deploying system to target platform, exposing required network interfaces, fixing accidental mistakes.
    \item Testing and production launch: Process of manual testing of the ready system by developers, making the feature available to public.
    \item Monitoring and maintenance: Ongoing activities to monitor the system's performance, availability, and security as well as identify the bugs and fix them.
    \item Feedback and issue solving: The process of gathering user feedback, addressing reported issues, deciding on important user needs and deriving new requirements for quiz subsystem.
\end{enumerate}

\subsection{Outsourced tasks}

None of the tasks will be completely outsourced, however a subset of listed tasks will be:

\begin{itemize}
    \item User interface plan: Team does not have experience with voice assistant interface design, that will be done by external team.
    \item User interface plan: Mobile design will be done by external team.
    \item Error analysis: Part of manual testing will be outsourced to external team.
    \item Testing and production launch: External team will be responsible for live testing the application.
\end{itemize}

\subsection{Gantt chart}

\begin{figure}[ht]
    \centering
    \includegraphics[width=\textwidth]{../lab3diags/Gantt_image.png}
    \caption{Gantt chart}
    \label{gantt-table}
\end{figure}

\section{Traceability}

\begin{figure}[htbp]
    \centering
    \label{tab:domain_use_cases}
    \begin{tabular}{|l|*{4}{c|}}
    \hline
    \backslashbox{Entity}{Use Case} & \makecell{List \\ Quizzes} & \makecell{Take \\ Quiz} & \makecell{Delete \\ Quiz} & \makecell{Create \& Edit \\ Quiz} \\ \hline
    User          & X & X & & \\ \hline
    Topic Owner   & X & & X & X \\ \hline
    Topic         & X & & X & X \\ \hline
    Quiz          & X & X & X & X \\ \hline
    Question      & & X & & X \\ \hline
    Option        & & X & & X \\ \hline
    Quiz Session  & & X & & \\ \hline
    User Answer   & & X & & \\ \hline
    \end{tabular}
    \caption{Domain entities to use cases traceability table}
\end{figure}
    
\begin{figure}[htbp]
    \centering
    \label{tab:requirements_use_cases}
    \begin{tabular}{|l|*{7}{c|}}
    \hline
    \backslashbox{Requirement}{Use Case} & \makecell{List \\ Quizzes} & \makecell{Take \\ Quiz} & \makecell{Delete \\ Quiz} & \makecell{Create \\ Quiz} & \makecell{Edit \\ Quiz} & \makecell{Quiz \\ Editor}\\\hline
    FR1  & X & & & & & \\ \hline
    FR2  & X & X & & & & \\ \hline
    FR3  & & & X & & & X \\ \hline
    FR4  & & X & & & & \\ \hline
    FR5  & & X & & & & \\ \hline
    FR6  & & X & X & & & X \\ \hline
    FR7  & & & X & & & \\ \hline
    FR8  & & & & X & & \\ \hline
    FR9  & & & & & X & \\ \hline
    FR10 & & & X & & & \\ \hline
    FR11 & & & & & & X \\ \hline
    FR12 & & & & & & X \\ \hline
    FR13 & & & X & & & X \\ \hline
    NFR1 & & X & & & & X \\ \hline
    NFR2 & X & X & X & & & X \\ \hline
    NFR3 & X & X & X & & & X \\ \hline
    NFR4 & & X & & & & X \\ \hline
    \end{tabular}
    \caption{Requirements to use cases traceability table}
\end{figure}

\listoffigures

\end{document}
