\documentclass[
    english, % Klasei padavus parametrą 'english', darbas bus anglų kalba.
    % signatureplaces % prideda parašų vietas tituliniame puslapyje
]{VUMIFPSkursinis}
\usepackage{float}
\usepackage{wrapfig2}
\usepackage{hyperref}
\usepackage{algorithmicx}
\usepackage{algorithm}
\usepackage{algpseudocode}
\usepackage{amsfonts}
\usepackage{amsmath}
\usepackage{bm}
\usepackage{caption}
\usepackage{color}
\usepackage{graphicx}
\usepackage{listings}
\usepackage{subcaption}
\usepackage{biblatex}
\usepackage{geometry}
\usepackage{booktabs}
\usepackage{multirow}
\usepackage{diagbox}
\usepackage{afterpage}
\usepackage{makecell}
\usepackage[inkscapelatex=false]{svg}
\renewcommand{\cftdotsep}{1} 


% Titulinio aprašas
\university{Vilnius university}
\faculty{Faculty of mathematics and informatics}
\department{Software engineering study program}
\papertype{Software Engineering II laboratory work 2}
\title{EduPal change request analysis, requirements and project plan}
\status{2 course 5 group students}
\author{Motiejus Šveikauskas}
\secondauthor{Kanstantinas Piatrashka}
\thirdauthor{Aldas Vertelis}
\fourthauthor{Danielius Podbielski}
\reviewer{doc. dr. Vardauskas Pavardauskas}
\date{Vilnius – \the\year}

\bibliography{bibliografija}

\begin{document}
\maketitle

\tableofcontents

\section{Context}

EduPal is a learning aid business which helps both students and teachers by allowing educators to conveniently upload course information for students to learn. EduPal offers unique features such as pomodoro timer, note creation and export in pdf, setting goals, chatGPT integration. EduPal lets users have subjects, create topics inside subjects and upload conspectuses.

\subsection{Change request}

\textbf{User need:} Students want to revise learned material in interactive and efficient way. We propose an idea to integrate quizzes into the system. Every topic should have a quiz section where topic owner  should be able to create, update and delete quizzes. Quiz consists of randomly shuffled questions, question answer interface should be simple and easy to use with mouse or touch device. Upon answering a question, student gets an instant feedback.

\vspace{\baselineskip}

Currently EduPal system offers no way to revise or test knowledge of a particualr topic. It is responsibility of users to check their own understanding of material by using their own aproach which takes more time than they would like and adds a complexity of choosing a third party service or methodology. Integrating quizzes into a system would boost engagement of users and simplify their studying process.

\subsection{Change priority}

Majority of students need to use EduPal with third party active learning tool that provides feedback (typically flashcard or quiz platform). Many major players (such as Quizlet and Kahoot) do not yet have proper conspectus storage system such as EduPal, meaning that users of those platforms tend to rely on solutions such as EduPal. Implementing quizzes can motivate active learning tool users to try EduPal while some may decide that only EduPal is enough for their needs. Additionaly, active learning tool platforms may implement data storage and sharing system similar to EduPal that would result in users migrating away from EduPal. Benefits of built in quiz capability include:

\begin{enumerate}
    \item EduPal can be treated as all-in-one system
    \item Users will not need to search for and configure third party systems
    \item EduPal can be advertised to a broader user base.
    \item Quizzes would also provide valuable data insights into user engagement and performance, which could be used for strategic decisions by managers.
\end{enumerate}

And drawbacks are:

\begin{enumerate}
    \item Being all-in-one system, EduPal will have to compete with both conspectus storage systems and active learning tool systems which requires more financial investements
    \item Some users will not want to use EduPal solution as they are acustomed to their setups
\end{enumerate}

We consider this change priority to be \textbf{medium to high}.

\subsection{Affected stakeholders}

The feature would affect these stakeholders:

\begin{enumerate}
    \item Teachers would get the opportunity to emphasize more important questions by creating quizzes and would have the chance to provide more helpful material in a different format.
    \item Students would have the opportunity to learn in an easier and more useful way. The opportunity to test their own knowledge would allow students to realize which topics they know well and on which topics they should revise.
    \item Developers and Maintainers would have to create, test and maintain the feature.
    \item Marketers would have to advertise the feature.
    \item Business owners and Investors will likely see increase in user base and profits.
\end{enumerate}

\subsection{Outsourcing capability}

Although outsourcing the feature to other developers might help save time and be an easier option, we decided to build the feature in house. We would have complete control of how our feature would look and work. Also, this way we would avoid any dependencies on third-party services, maintain current application efficiency, and save capital.

\subsection{Implementation alternatives}

\subsubsection{First option: use notes to provide links to third party quiz service}

The simplest implementation alternative is also one that required no implementation whatsoever. Our suggestion is to use the already existing note feature to add a link that would take whoever wishes to test their knowledge to a third-party quiz app. There, a new user account would have to be created if the student is using the app for the first time, otherwise logging in is sufficient. Once authenticated the student will be presented with the quiz they wished to take. The price to implement this option will not cost our client anything if they find the suitable third-party app themselves, otherwise, if we are tasked with determining the most suitable third-party app, this implementation will cost a trivial sum. To complete this option would take very little time and virtually no qualifications, as all there is already a way to leave a message, link for a topic. The disadvantages of this approach are that both the teachers and the students will be forced to create a new account in a separate application, if any sort of progress, score tracking is available, it will stay separately in the third-party app, however the benefit of this approach is that it can be made readily available in a matter of hours and at virtually no cost. Cost for this implementation is minimal and possibly costs nothing if the user finds a suitable third-party application. This option aligns with our vision to create an effective learning environment and empower individuals on their path of continuous improvement, but differs from our mission to offer services for individual, in this case we would be offering a third-party service, which would be out of our control. However this option is good looking from a business standpoint: firstly, cost is minimal, secondly requires no qualifications for implementing this solution and lastly, this solution can be implemented within minutes, thus saving time for implementing harder features. But there are downsides as users may be discouraged by the use of third-party software and potential less than ideal user experience.

\subsubsection{Second option: integrate third-party API}

The second implementation alternative we were able to define is a middle ground between the first and the third option detailed below. Our team proposes integrating a third-party quizzing app into our system through an API, perhaps even establishing a potential partnership with a suitable quizzing app. This implementation would feature a button on every topic which would take users to a new tab or a window and immediately begin the quiz with no authorization required. If needed, quiz scores could be stored in our system and displayed to the user. This approach would be far more convenient for everyone on the platform and users would be able to enjoy a feature-rich quizzing experience, however there are a few disadvantages. Users would have to deal with different stylistic choices and system owners may have concerns for data privacy and performance which we will not be able to control. We would require a designer, 1 or 2 software engineers and a tester working on this feature for 2 – 3 weeks. Thus the budget for this implementation will have to be relatively bigger and the third party quiz API could incur more costs if it isn’t free. This option aligns with our vision and our mission alongside it. Different from the first option is that we are able to offer services partly controlled by us, thus offering more credibility and trust to our clients. Looking from a business standpoint this implementation is not ideal because of considerable costs for implementing a solution which requires back-end developers to work with a third-party API, also front-end developers are required to implement a pleasant user experience while having testers to ensure this feature works as intended. And lastly, there may be added costs if the API is not free and we are not able to establish a partnership with another service provider.

\subsubsection{Third option: add quiz subsystem to EduPal}

Our team implements a fully fledged quizzing feature in the EduPal system, allowing much easier use for everyone. The cost for this option is considerably greater than the other two previously mentioned because we would have to assign a considerable amount of developers and testers, possibly a full-fledged team would have to be assigned for this task as it comes as a separate service. However this option aligns with our vision and mission perfectly because we would be creating an effective learning environment while also having the ability to improve the feature, and we would be offering our personal services to users and in this way maintaining full control of the application. Looking from a business standpoint this implementation is good for a few reasons. Mainly having our own quizzing service increases the credibility of our software and encourages clients to use EduPal over other alternatives. Secondly, expanding this feature and providing feedback, fixing bugs would be much easier and less time consuming as opposed to asking other service providers to fix our issues or even not having them fixed at all. While it does come at a considerable price, in the long run this option is better than the two previously mentioned.

\section{Domain model}

\begin{figure}[ht]
    \centering
    \includegraphics[width=\textwidth]{../lab3diags/domain.drawio.png}
    \caption{Domain model diagram}
    \label{domain-model}
\end{figure}

\begin{description}
    \item[User] Every EduPal customer is user - they are able to create subjects, topics, upload conspectuses, rate them and so on.
    \item[Topic owner] User who owns a particular topic - only they are able to edit its contents, including quizzes.
    \item[Topic] A part of subject (not related to change request). Topic consists of conspectuses, comments and quizzes.
    \item[Quiz] A named collection of questions
    \item[Question] Question has a name and a set of incorrect options as well as correct one.
    \item[Option] Question option is a text answer.
    \item[Quiz session] Users start quiz session when they take the quiz.
    \item[Quiz session question] Questions are shufled during quiz session.
    \item[Question feedback] Feedback whether question answer is correct or not.
    \item[User answer] User answer is an option that was picked by user.
\end{description}

\section{Use cases}

\subsection{List quizzes}

\noindent\textbf{\fontsize{13}{15}\selectfont Main scenario:}

User opens EduPal website; system displays login page. User enters login information and presses Login button; system verifies identity of user, starts user session and shows Subject list page. User clicks on specific subject; system loads its topics and show Topic list page for that subject. User clicks on a specific topic; system loads topic data and shows Topic page for that topic. User clicks on Quizzes button; system fetches a list of quizzes associated with that topic, checks whether user is a topic owner and shows a Quiz list page. User clicks on a particular quiz; system displays a Quiz page.

\noindent\textbf{\fontsize{13}{15}\selectfont Alternatives:}

\textbf{User enters incorrect login information:} System displays Login failed message.

\textbf{User session expired:} System navigates user to Login page.

\textbf{User is a topic owner:} System shows Create quiz button in Quiz list page. Systems shows edit and delete buttons in Quiz page.

\subsection{Create quiz}

\noindent\textbf{\fontsize{13}{15}\selectfont Main scenario:}

Topic owner lists quizzes and clicks Create quiz button; system shows Quiz editor page. Topic owner enters quiz name in Quiz name field; system checks quiz name validity. Topic owner clicks add question button; system adds question editor component to Quiz editor page with 2 empty options, with first option marked as correct answer. Topic owner enters question name in question name field and option names in option name fields; system checks validity of entered information. Topic owner clicks Add option button; system adds a new option input to question editor. Topic owner drags the option they want to be correct to top of the question option list; system marks dragged option as correct option. Topic owner clicks on Create quiz button; system saves the quiz and shows quiz page for saved quiz.

\noindent\textbf{\fontsize{13}{15}\selectfont Alternatives:}

\textbf{Bad input: Quiz name is shorter than 5 characters or longer than 50:}:System shows name error message near quiz name field.

\textbf{Bad input: Quiz has no questions:} System shows No questions text near Add question button.

\textbf{Bad input: Question or option name is shorter than 5 characters or longer than 50:} System show name error message near affected fields.

\textbf{There are bad inputs:} System disables Create quiz button.

\textbf{Topic owner wants to add question:} Topic owner clicks on add question button; system adds new question editor component.

\textbf{Topic owner wants to delete question:} Topic owner clicks on delete question button; system deletes a question and removes question editor component associated with that question.

\textbf{Topic owner wants to add option to question:} Topic owner clicks on Add option button; system adds new empty option at the end of option list.

\textbf{Topic owner want to remove option:} Topic owner clicks on Remove option button; system removes the option. If option was marked as the correct one (was on top of the list), second option becomes correct.

\textbf{There are 4 options:} System does not show Add option button.

\textbf{There are 2 optionsL} System does not show Remove option buttons.

\textbf{Submitted quiz has bad inputs or questions with more than 5 or less than 2 options:} System aborts the request and displays Something wrong happenned popup.

\subsection{Take a quiz}

\noindent\textbf{\fontsize{13}{15}\selectfont Main scenario:}

User lists quizzes and clicks on a particular quiz, system fetches quiz data and shows Quiz page. User clicks Play button; system creates a quiz session with the user, shuffles the questions and displays question page. User clicks on option; system records an answer and show next question. After there are no more unanswered session questions, system calculates the percentage of correctly answered questions and shows quiz completed page.

\noindent\textbf{\fontsize{13}{15}\selectfont Alternatives:}

\textbf{User answer was correct:} System displays correct answer notification for 5 seconds.

\textbf{Uswer answer was wrong:} System displays a wrong answer page. User clicks on continue button; system proceeds with the next question.

\textbf{User clicks on restart button on quiz completed page:} System reshufles questions and starts quiz session anew.

\textbf{User leaves question page or clicks on exit button on quiz completed page:} System deletes information about the session and shows a quiz page.

\subsection{Edit quiz}

TODO: make sure to extract quiz editor interactions so that create quiz and edit quiz are similar.

\subsection{Delete quiz}

\noindent\textbf{\fontsize{13}{15}\selectfont Main scenario:}

Topic owner lists quizzes and clicks on a particular quiz; system fetches quiz data and displays quiz page. Topic owner clicks on Delete quiz button; system shows deletion confirmation popup. Topic owner click on Confirm deletion button; system deletes, shows quiz list page for a topic that deleted quiz belonged and displays Quiz deleted notification.

\noindent\textbf{\fontsize{13}{15}\selectfont Alternatives:}

\textbf{Subject owner closes confirmation popup:} System aborts deletion.

TODO: alternatives when there are active quiz sessions.

\listoffigures
\printbibliography[heading=bibintoc]

\end{document}
