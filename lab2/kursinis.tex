\documentclass[
    english, % Klasei padavus parametrą 'english', darbas bus anglų kalba.
    % signatureplaces % prideda parašų vietas tituliniame puslapyje
]{VUMIFPSkursinis}
\usepackage{float}
\usepackage{wrapfig2}
\usepackage{hyperref}
\usepackage{algorithmicx}
\usepackage{algorithm}
\usepackage{algpseudocode}
\usepackage{amsfonts}
\usepackage{amsmath}
\usepackage{bm}
\usepackage{caption}
\usepackage{color}
\usepackage{graphicx}
\usepackage{listings}
\usepackage{subcaption}
\usepackage{biblatex}
\usepackage[inkscapelatex=false]{svg}
\renewcommand{\cftdotsep}{1} 


% Titulinio aprašas
\university{Vilnius university}
\faculty{Faculty of mathematics and informatics}
\department{Software engineering study program}
\papertype{Software Engineering II laboratory work 2}
\title{EduPal system architecture}
\status{2 course 5 group students}
\author{Motiejus Šveikauskas}
\secondauthor{Kanstantinas Piatrashka}
\thirdauthor{Aldas Vertelis}
\fourthauthor{Danielius Podbielski}
\reviewer{doc. dr. Vardauskas Pavardauskas}
\date{Vilnius – \the\year}

\bibliography{bibliografija}

\begin{document}
\maketitle

\tableofcontents

\section{Context}
EduPal is a learning aid system designed to enhance
the educational experience for both students and teachers. With a user-friendly interface and a comprehensive set of features, EduPal aims to streamline the learning process, fostering a more efficient and engaging educational environment.

\subsection{Functionality overview}
List features here

\subsection{Tech stack}
List tech stack here

\subsection{Selected requirements}
In this document we will focus on system changes needed to fulfil the follwing set of requirements:

\begin{itemize}
    \item \textbf{FR1.} Subject owner shall be able to archive and unarchive the subject. Archived subject does not support adding, deleting or changing topics and uploaded conspectuses.
    \item \textbf{FR2.} System shall automatically archive subjects with no recent topic or conspectus changes after 1 month.
    \item \textbf{FR3.} Subject owner shall be notified about automatic archivation.
    \item \textbf{FR4.} Users and administrators shall be able to be notified via system notification and email.
    \item \textbf{FR5.} Administrator shall be able create other adminstrator accounts.
    \item \textbf{FR6.} Administrator actions shall be logged.
    \item \textbf{FR7.} Administrator shall be able to access OpenAI service usage statistics.
    \item \textbf{FR8.} Administrator shall be able to set OpenAI usage limits.
    \item \textbf{FR9.} System shall notify administrator if OpenAI usage exceeds limits.
    \item \textbf{FR10.} Users shall be able to leave feedback that administrators can access. Administrators shall be notified.
    \item \textbf{FR11.} Only user who created the subject shall be able to change subject metadata, create, update or delete new topics and conspectuses.
\end{itemize}

\section{Logical View}

\subsection{System models}

\begin{figure}[ht]
    \centering
    \includesvg[width=\textwidth]{../lab2diags/logical-view/models.drawio.svg}
    \label{system-models}
    \caption{System models}
\end{figure}

\begin{description}
    \item[Subjects and topics] Subject is a primary organizational unit of the system. Subject is owned by user. Inside subjects there are topics. Last update date is used by automatic archivation process.
    \item[Conspectuses] Conspectuses are files stored in topics.
    \item[Users] User model store login and contact information. Users are able to specificy knowledge level of a particular topic.
    \item[Comments] Users are able to post text comments on topics
    \item[Goals] User can have multiple goals all of which are linked to a particular set of subjects.
    \item[Notifications] Notifications are sent to user and can be accessed in the UI. Subclasses are used to provide additional functionality, such as linking subject, so UI can display a button that redirects to subject page or offers a different set of controls.
    \item[ChatMessages] ChatMessages are stored for OpenAI chat.
    \item[Notes] Every user can have global text notes.
    \item[Pomodoro] User has a single pomodoro session that they can tweak and control.
    \item[Administrators] Administrators are detached from the system and have their own views and controls.
    \item[AdministratorActions] System stores different types of administrator actions to ensure that administrators are responsible.
    \item[OpenAIStats] Object of this class is used by system to store token usage by day and it is sinced with data from OpenAI servers from time to time.
\end{description}

\section{Development View}

\section{Process View}

\section{Physical View}

\section{Use Case View}

\subsection{Subject archivation}

\subsubsection{Archive and unarchive subject}

\begin{figure}[ht]
    \centering
    \includesvg{../lab2diags/use-cases/archivation.drawio.svg}
    \label{archivation-use-case}
    \caption{Subject archivation and unarchivation use case diagram}
\end{figure}

\begin{description}
    \item[Actors] User
    \item[Requirements] FR1, FR11
    \item[Description] User can request subject archivation or unarchivation. In both cases system must check whether user is the owner of a particular subject. If that is the case, subjcet archivation state is updated, otherwise nothing happens.
\end{description}

\subsubsection{Automatic subject archivation}

\begin{figure}[ht]
    \centering
    \includesvg{../lab2diags/use-cases/automatic-archivation.drawio.svg}
    \label{automatic-archivation-use-case}
    \caption{Automatic subject archivation diagram}
\end{figure}

\begin{description}
    \item[Actors] User, System clock
    \item[Requirements] FR2, FR3, FR11
    \item[Description] System clock regularly initates automatic archivation check process. All users who own subjects that were not updated in a specified amount of time will be notified about archivation. If they do not respond, their subject will be automatically archived.
\end{description}

\subsection{Administration}

\begin{figure}[ht]
    \centering
    \includesvg[width=\textwidth]{../lab2diags/use-cases/administration.drawio.svg}
    \label{administration-use-cases}
    \caption{Administration use cases}
\end{figure}

\subsubsection{Administrator account creation}

\begin{description}
    \item[Actors] Administrator
    \item[Requirements] FR4, FR5
    \item[Description] Administrator can create other administrator accounts. This action is logged.
\end{description}

\subsubsection{OpenAI usage control}

\begin{description}
    \item[Actors] Administrator
    \item[Requirements] FR5, FR6, FR7
    \item[Description] Administrator can view OpenAI usage (used tokens, price) and set limits. Action of setting limits is logged.
\end{description}

\subsubsection{Notification on OpenAI usage}

\begin{description}
    \item[Actors] Administrator, User
    \item[Requirements] FR8
    \item[Description] Every time user uses OpenAI integration, system calculates usage and notifies administrator if it exceeds limits.
\end{description}

\subsubsection{Feedback collection}

\begin{description}
    \item[Actors] Administrator, User
    \item[Requirements] FR9
    \item[Description] Users can leave feedback that administrators have access to.
\end{description}

\section{Traceability}

\listoffigures
\printbibliography[heading=bibintoc]

\end{document}
