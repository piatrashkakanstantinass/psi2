\documentclass[
    english, % Klasei padavus parametrą 'english', darbas bus anglų kalba.
    % signatureplaces % prideda parašų vietas tituliniame puslapyje
]{VUMIFPSkursinis}
\usepackage{float}
\usepackage{wrapfig2}
\usepackage{hyperref}
\usepackage{algorithmicx}
\usepackage{algorithm}
\usepackage{algpseudocode}
\usepackage{amsfonts}
\usepackage{amsmath}
\usepackage{bm}
\usepackage{caption}
\usepackage{color}
\usepackage{graphicx}
\usepackage{listings}
\usepackage{subcaption}
\usepackage{biblatex}
\usepackage[inkscapelatex=false]{svg}


% Titulinio aprašas
\university{Vilnius university}
\faculty{Faculty of mathematics and informatics}
\department{Software engineering study program}
\papertype{Software Engineering II laboratory work 2}
\title{EduPal system architecture}
\status{2 course 5 group students}
\author{Motiejus Šveikauskas}
\secondauthor{Kanstantinas Piatrashka}
\thirdauthor{Aldas Vertelis}
\fourthauthor{Danielius Podbielski}
\reviewer{doc. dr. Vardauskas Pavardauskas}
\date{Vilnius – \the\year}

\bibliography{bibliografija}

\begin{document}
\maketitle

\tableofcontents

\section{Context}
EduPal is a learning aid system designed to enhance
the educational experience for both students and teachers. With a user-friendly interface and a comprehensive set of features, EduPal aims to streamline the learning process, fostering a more efficient and engaging educational environment.

\subsection{Functionality overview}
List features here

\subsection{Tech stack}
List tech stack here

\subsection{Selected requirements}
In this document we will focus on system changes needed to fulfil the follwing set of requirements:

\begin{itemize}
    \item \textbf{FR1.} User shall be able to archive\footnote{Archived subject is a read-only subject} the subject.
    \item \textbf{FR2.} User shall be able to unarchive the subject.
    \item \textbf{FR3.} System shall automatically archive subjects with no recent changes.
    \item \textbf{FR4.} Users shall be notified about automatic archivation.
    \item \textbf{FR5.} Administrator shall be able create other adminstrator accounts.
    \item \textbf{FR6.} Administrator actions shall be logged.
    \item \textbf{FR7.} Administrator shall be able to access OpenAI service usage statistics.
    \item \textbf{FR8.} Administrator shall be able to set OpenAI usage limits.
    \item \textbf{FR9.} System shall notify administrator if OpenAI usage exceeds limits.
    \item \textbf{FR10.} Users shall be able to leave feedback.
    \item \textbf{FR11.} Administrators shall be able to access user feedback.
    \item \textbf{NR1.} System shall handle heavy load.
\end{itemize}

\section{Use Case View}

\includesvg{../lab2diags/use-cases/archivation.drawio.svg}
weawdw
\section{Logical View}

\section{Development View}

\section{Process View}

\section{Physical View}

\section{Traceability}

\printbibliography[heading=bibintoc]

\end{document}
