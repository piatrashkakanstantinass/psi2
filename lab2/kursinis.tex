\documentclass[
    english, % Klasei padavus parametrą 'english', darbas bus anglų kalba.
    % signatureplaces % prideda parašų vietas tituliniame puslapyje
]{VUMIFPSkursinis}
\usepackage{float}
\usepackage{wrapfig2}
\usepackage{hyperref}
\usepackage{algorithmicx}
\usepackage{algorithm}
\usepackage{algpseudocode}
\usepackage{amsfonts}
\usepackage{amsmath}
\usepackage{bm}
\usepackage{caption}
\usepackage{color}
\usepackage{graphicx}
\usepackage{listings}
\usepackage{subcaption}
\usepackage{biblatex}
\usepackage[inkscapelatex=false]{svg}
\renewcommand{\cftdotsep}{1} 


% Titulinio aprašas
\university{Vilnius university}
\faculty{Faculty of mathematics and informatics}
\department{Software engineering study program}
\papertype{Software Engineering II laboratory work 2}
\title{EduPal system architecture}
\status{2 course 5 group students}
\author{Motiejus Šveikauskas}
\secondauthor{Kanstantinas Piatrashka}
\thirdauthor{Aldas Vertelis}
\fourthauthor{Danielius Podbielski}
\reviewer{doc. dr. Vardauskas Pavardauskas}
\date{Vilnius – \the\year}

\bibliography{bibliografija}

\begin{document}
\maketitle

\tableofcontents

\section{Context}
EduPal is a learning aid system designed to enhance
the educational experience for both students and teachers. With a user-friendly interface and a comprehensive set of features, EduPal aims to streamline the learning process, fostering a more efficient and engaging educational environment.

\subsection{Functionality overview}
List features here

\subsection{Tech stack}
List tech stack here

\subsection{Selected requirements}
In this document we will focus on system changes needed to fulfil the follwing set of requirements:

\begin{itemize}
    \item \textbf{FR1.} User shall be able to archive and unarchive\footnote{Archived subject is a read-only subject} the subject.
    \item \textbf{FR2.} System shall automatically archive subjects with no recent changes.
    \item \textbf{FR3.} Users shall be notified about automatic archivation.
    \item \textbf{FR4.} Administrator shall be able create other adminstrator accounts.
    \item \textbf{FR5.} Administrator actions shall be logged.
    \item \textbf{FR6.} Administrator shall be able to access OpenAI service usage statistics.
    \item \textbf{FR7.} Administrator shall be able to set OpenAI usage limits.
    \item \textbf{FR8.} System shall notify administrator if OpenAI usage exceeds limits.
    \item \textbf{FR9.} Users shall be able to leave feedback.
    \item \textbf{FR10.} Administrators shall be able to access user feedback.
    \item \textbf{FR11.} Only user who created the subject shall be able to change its contents, request archivation and handle other changes.
    \item \textbf{NR1.} System shall handle heavy load.
\end{itemize}

\section{Use Case View}

\subsection{Subject archivation}

\subsubsection{Archive and unarchive subject}

\begin{figure}[ht]
    \centering
    \includesvg{../lab2diags/use-cases/archivation.drawio.svg}
    \label{archivation-use-case}
    \caption{Subject archivation and unarchivation use case diagram}
\end{figure}

\begin{description}
    \item[Actors] User
    \item[Requirements] FR1, FR11
    \item[Description] User can request subject archivation or unarchivation. In both cases system must check whether user is the owner of a particular subject. If that is the case, subjcet archivation state is updated, otherwise nothing happens.
\end{description}

\subsubsection{Automatic subject archivation}

\begin{figure}[ht]
    \centering
    \includesvg{../lab2diags/use-cases/automatic-archivation.drawio.svg}
    \label{automatic-archivation-use-case}
    \caption{Automatic subject archivation diagram}
\end{figure}

\begin{description}
    \item[Actors] User, System clock
    \item[Requirements] FR2, FR3, FR11
    \item[Description] System clock regularly initates automatic archivation check process. All users who own subjects that were not updated in a specified amount of time will be notified about archivation. If they do not respond, their subject will be automatically archived.
\end{description}

\subsection{Administration}

\begin{figure}[ht]
    \centering
    \includesvg[width=\textwidth]{../lab2diags/use-cases/administration.drawio.svg}
    \label{administration-use-cases}
    \caption{Administration use cases}
\end{figure}

\subsubsection{Administrator account creation}

\begin{description}
    \item[Actors] Administrator
    \item[Requirements] FR4, FR5
    \item[Description] Administrator can create other administrator accounts. This action is logged.
\end{description}

\subsubsection{OpenAI usage control}

\begin{description}
    \item[Actors] Administrator
    \item[Requirements] FR5, FR6, FR7
    \item[Description] Administrator can view OpenAI usage (used tokens, price) and set limits. Action of setting limits is logged.
\end{description}

\subsubsection{Notification on OpenAI usage}

\begin{description}
    \item[Actors] Administrator, User
    \item[Requirements] FR8
    \item[Description] Every time user uses OpenAI integration, system calculates usage and notifies administrator if it exceeds limits.
\end{description}

\subsubsection{Feedback collection}

\begin{description}
    \item[Actors] Administrator, User
    \item[Requirements] FR9
    \item[Description] Users can leave feedback that administrators have access to.
\end{description}

\section{Logical View}

\section{Development View}

\section{Process View}

\section{Physical View}

\subsection{Deployment Diagram}

The diagram below showcases how the deployed containers look like in the Azure Kubernetes Cluster.

\begin{figure}[ht]
    \centering
    \includesvg[width=\textwidth]{../lab2diags/PhysicalView/deployment.svg}
    \label{deployment-showcase}
    \caption{Deployment showcase}
\end{figure}

\subsection{Resources}

\textbf{Providers:}
\begin{itemize}
    \item Cloud Provider: Microsoft Azure
    \item CI/CD Platform: Gitlab
\end{itemize}

\textbf{Reasons for choosing Azure:}
\begin{itemize}
    \item Free Tier Availability: Azure offers a generous free tier that allows you to experiment and deploy your application without initial costs.
    \item Scalability: Azure offers a wide range of services and resources that can easily scale up or down based on your application's needs.
    \item Integration with GitLab: While Azure DevOps is a native option, GitLab integrates well with Azure for deployments and managing resources.
\end{itemize}

\textbf{Plan/Tier:}
\begin{itemize}
    \item Azure Free Trial: Free tier offered by Azure provides access to various services with limitations on compute hours, storage, etc. This is sufficient for development and initial deployment.
\end{itemize}

\textbf{Database Management System:}
\begin{itemize}
    \item Azure SQL Database: This is a managed relational database service offered by Azure. It provides a familiar SQL experience with automatic scaling, high availability, and built-in security features.
\end{itemize}

\textbf{Security Considerations:}
\begin{itemize}
    \item Secure Coding Practices: Implement secure coding practices in both the React app and .NET backend to prevent common vulnerabilities.
    \item Secret Management: Store sensitive information like database connection strings and API keys as environment variables or app settings in Azure App Service. This keeps sensitive information separate from the code and helps with managing different environments.
    \item Network Security Groups: Use NSGs to restrict inbound and outbound traffic to your Azure resources, only allowing access from authorized sources.
    \item Regular Updates: Keep your application code, libraries, and frameworks updated with the latest security patches to address known vulnerabilities.
\end{itemize}

\textbf{Operating System:}
\begin{itemize}
    \item Linux Images:
        \begin{itemize}
            \item Cost-Effective: Linux typically has lower licensing costs compared to Windows Server.
            \item Open-Source Community: Linux benefits from a large and active open-source community, providing extensive support and resources.
            \item Lightweight: Linux distributions are generally lightweight which can improve performance, especially for containerized deployments.
        \end{itemize}
\end{itemize}

\textbf{Scaling Strategy:}
\begin{itemize}
    \item Vertical Scaling: This involves upgrading the resources allocated to your existing Azure App Service instances (e.g., increasing CPU cores or memory). This can improve performance without adding additional instances.
\end{itemize}

\subsection{Deployment process}

The deployment process involves the following steps:

\begin{enumerate}
    \item User develops a frontend React application and an ASP.NET API with a PostgreSQL database.
    \item User pushes code to GitLab repository.
    \item The CI pipeline tests the code (Unit, Integration tests).
    \item The CI pipeline publishes artifacts to Azure Container Registry.
    \item The CD pipeline pulls artifact images from Azure Container Registry.
    \item The CD pipeline deploys the web application on the Azure cloud.
\end{enumerate}

\section{Traceability}

\listoffigures
\printbibliography[heading=bibintoc]

\end{document}
