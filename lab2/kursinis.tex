\documentclass[
    english, % Klasei padavus parametrą 'english', darbas bus anglų kalba.
    % signatureplaces % prideda parašų vietas tituliniame puslapyje
]{VUMIFPSkursinis}
\usepackage{float}
\usepackage{wrapfig2}
\usepackage{hyperref}
\usepackage{algorithmicx}
\usepackage{algorithm}
\usepackage{algpseudocode}
\usepackage{amsfonts}
\usepackage{amsmath}
\usepackage{bm}
\usepackage{caption}
\usepackage{color}
\usepackage{graphicx}
\usepackage{listings}
\usepackage{subcaption}
\usepackage{biblatex}
\usepackage[inkscapelatex=false]{svg}
\renewcommand{\cftdotsep}{1} 


% Titulinio aprašas
\university{Vilnius university}
\faculty{Faculty of mathematics and informatics}
\department{Software engineering study program}
\papertype{Software Engineering II laboratory work 2}
\title{EduPal system architecture}
\status{2 course 5 group students}
\author{Motiejus Šveikauskas}
\secondauthor{Kanstantinas Piatrashka}
\thirdauthor{Aldas Vertelis}
\fourthauthor{Danielius Podbielski}
\reviewer{doc. dr. Vardauskas Pavardauskas}
\date{Vilnius – \the\year}

\bibliography{bibliografija}

\begin{document}
\maketitle

\tableofcontents

\section{Context}
EduPal is a learning aid system designed to enhance
the educational experience for both students and teachers. With a user-friendly interface and a comprehensive set of features, EduPal aims to streamline the learning process, fostering a more efficient and engaging educational environment.

\subsection{Functionality overview}

At present, EduPal has these functionalities:

\begin{itemize}
    \item \textbf{Create and Manage Subjects}: Users have the ability to create new subjects and manage existing ones within the system. This includes creating topics under each subject and uploading notes in PDF format related to those topics.
    \item \textbf{Note Sharing and Interaction}: Users can interact with notes shared by others by viewing, rating, and engaging in discussions within the topic's section. The system allows users to rate notes based on usefulness to promote the most valuable ones. Users also have options to export notes to a PDF section, download them to their device, or save them as drafts for further editing. 
    \item \textbf{Pomodoro Timer System}: The system features a Pomodoro timer system designed to enhance productivity. Users can customize their study and break times according to their preferences. The timer notifies users when it's time to take a break, helping them maintain focus and manage their study sessions effectively.
    \item \textbf{Goal Setting}: Users can set study goals within the system, specifying desired subjects and study times. This feature enables users to plan and organize their study sessions more efficiently, helping them stay on track with their academic or learning objectives.
    \item \textbf{Integration with GPT API}: The project integrates with Chat GPT, providing users with additional functionalities. This integration allows the system to store user chat history and respond to user inquiries using the capabilities of Chat GPT. Users can leverage this feature to seek assistance, clarify doubts, or engage in interactive conversations within the system.
    \end{itemize}

\subsection{Tech stack and services}

The EduPal system was created using a modern technology stack to support its functionalities. The basic structure of the system:

\begin{itemize}
    \item \textbf{Frontend}: The frontend of the system is developed using the React framework and JavaScript with custom CSS. Frontend allows communication with the server via the fetch function, which utilizes the Fetch API to send HTTP requests to the particular API endpoint in the backend.
    \item \textbf{Backend}: The system is built using ASP.NET Core and uses Entity Framework Core which helps following monolithic architecture. It incorporates the Repository Pattern for handling data access, particularly for ASP.NET Identity user management. The backend is built using .NET 7.0 and follows a RESTful API architecture, enabling it to handle HTTP requests and responses. Additionally, WebSocket communication is facilitated by SignalR and is used for ChatHub endpoint which, communication via comment section. Passwords are stored directly as plaintext within the database. Logs are managed by the interceptor, tracking database changes. Also, tokens are generated upon successful user authentication. These tokens contain user information and are used for subsequent authentication and authorization requests between the client and the server.
    \item\textbf{Data Storage}: Conspectuses are stored locally within the application's file system. Upon upload, each conspectus is saved as a separate file in a directory named "Files" relative to the application's working directory. When users request to download a conspectus, the application retrieves the corresponding file from this directory and streams it back as downloadable content. Also PostgreSQL is used to store information about the users. The use of Entity Framework Core provides tighter integration with the backend for database operations.
    \item \textbf{OpenAI Service}: The system uses OpenAI API, to provide Chat GPT services. API helps connecting to the server and sending any user inqueries. The prompt sent to Open AI is modified so that length of the response would be reasonable.
    \item \textbf{Tests}: System is tested using a combination of unit and integration tests.
    \item \textbf{Version Control}: Git is used for code publishing and reviewing processes.
\end{itemize}

\subsection{Selected requirements}
In this document we will focus on system changes needed to fulfil the follwing set of requirements:

\begin{itemize}
    \item \textbf{FR1.} User shall be able to archive and unarchive\footnote{Archived subject is a read-only subject} the subject.
    \item \textbf{FR2.} System shall automatically archive subjects with no recent changes.
    \item \textbf{FR3.} Users shall be notified about automatic archivation.
    \item \textbf{FR4.} Administrator shall be able create other adminstrator accounts.
    \item \textbf{FR5.} Administrator actions shall be logged.
    \item \textbf{FR6.} Administrator shall be able to access OpenAI service usage statistics.
    \item \textbf{FR7.} Administrator shall be able to set OpenAI usage limits.
    \item \textbf{FR8.} System shall notify administrator if OpenAI usage exceeds limits.
    \item \textbf{FR9.} Users shall be able to leave feedback.
    \item \textbf{FR10.} Administrators shall be able to access user feedback.
    \item \textbf{FR11.} Only user who created the subject shall be able to change its contents, request archivation and handle other changes.
    \item \textbf{NR1.} System shall handle heavy load.
\end{itemize}

\section{Use Case View}

\subsection{Subject archivation}

\subsubsection{Archive and unarchive subject}

\begin{figure}[ht]
    \centering
    \includesvg{../lab2diags/use-cases/archivation.drawio.svg}
    \label{archivation-use-case}
    \caption{Subject archivation and unarchivation use case diagram}
\end{figure}

\begin{description}
    \item[Actors] User
    \item[Requirements] FR1, FR11
    \item[Description] User can request subject archivation or unarchivation. In both cases system must check whether user is the owner of a particular subject. If that is the case, subjcet archivation state is updated, otherwise nothing happens.
\end{description}

\subsubsection{Automatic subject archivation}

\begin{figure}[ht]
    \centering
    \includesvg{../lab2diags/use-cases/automatic-archivation.drawio.svg}
    \label{automatic-archivation-use-case}
    \caption{Automatic subject archivation diagram}
\end{figure}

\begin{description}
    \item[Actors] User, System clock
    \item[Requirements] FR2, FR3, FR11
    \item[Description] System clock regularly initates automatic archivation check process. All users who own subjects that were not updated in a specified amount of time will be notified about archivation. If they do not respond, their subject will be automatically archived.
\end{description}

\subsection{Administration}

\begin{figure}[ht]
    \centering
    \includesvg[width=\textwidth]{../lab2diags/use-cases/administration.drawio.svg}
    \label{administration-use-cases}
    \caption{Administration use cases}
\end{figure}

\subsubsection{Administrator account creation}

\begin{description}
    \item[Actors] Administrator
    \item[Requirements] FR4, FR5
    \item[Description] Administrator can create other administrator accounts. This action is logged.
\end{description}

\subsubsection{OpenAI usage control}

\begin{description}
    \item[Actors] Administrator
    \item[Requirements] FR5, FR6, FR7
    \item[Description] Administrator can view OpenAI usage (used tokens, price) and set limits. Action of setting limits is logged.
\end{description}

\subsubsection{Notification on OpenAI usage}

\begin{description}
    \item[Actors] Administrator, User
    \item[Requirements] FR8
    \item[Description] Every time user uses OpenAI integration, system calculates usage and notifies administrator if it exceeds limits.
\end{description}

\subsubsection{Feedback collection}

\begin{description}
    \item[Actors] Administrator, User
    \item[Requirements] FR9
    \item[Description] Users can leave feedback that administrators have access to.
\end{description}

\section{Logical View}

\subsection{System models}

\begin{figure}[ht]
    \centering
    \includesvg[width=\textwidth]{../lab2diags/logical-view/models.drawio.svg}
    \label{system-models}
    \caption{System models}
\end{figure}

\begin{description}
    \item[Subjects and topics] Subject is a primary organizational unit of the system. Subject is owned by user. Inside subjects there are topics. Last update date is used by automatic archivation process.
    \item[Conspectuses] Conspectuses are files stored in topics.
    \item[Users] User model store login and contact information. Users are able to specificy knowledge level of a particular topic.
    \item[Comments] Users are able to post text comments on topics
    \item[Goals] User can have multiple goals all of which are linked to a particular set of subjects.
    \item[Notifications] Notifications are sent to user and can be accessed in the UI. Subclasses are used to provide additional functionality, such as linking subject, so UI can display a button that redirects to subject page or offers a different set of controls.
    \item[ChatMessages] ChatMessages are stored for OpenAI chat.
    \item[Notes] Every user can have global text notes.
    \item[Pomodoro] User has a single pomodoro session that they can tweak and control.
    \item[Administrators] Administrators are detached from the system and have their own views and controls.
    \item[AdministratorActions] System stores different types of administrator actions to ensure that administrators are responsible.
    \item[OpenAIStats] Object of this class is used by system to store token usage by day and it is sinced with data from OpenAI servers from time to time.
\end{description}

\section{Development View}

\section{Process View}

\section{Physical View}

\section{Traceability}

\listoffigures
\printbibliography[heading=bibintoc]

\end{document}
